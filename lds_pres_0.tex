\documentclass[11pt]{beamer}
% \DeclareMathAlphabet      {\mathbfit}{OML}{cmm}{b}{it}
% use berlin theme
\usepackage{amsmath}
% \usepackage{unicode-math} 
\usetheme{Berlin}
\begin{document}
% \date{\today}
\title[Laplacian LDS]{The autonomous reaction-diffusion LDS}
% \subtitle{Subtitle (if any)}
\author[Andrei Sultan]{Andrei Sultan}
\institute[Institutul de Matematică şi Informatică "Vladimir Andrunachievici"]{Institutul de Matematică şi Informatică "Vladimir Andrunachievici"}
\date{\today} % You can replace \today with a specific date if needed
\begin{frame}
    \titlepage
\end{frame}

\begin{frame}{Outline}
    \tableofcontents
\end{frame}

\section{Introduction}
\subsection{Basic definitions}

\begin{frame}
    Let $(\mathfrak{X}, \mathnormal{d})$ be a complete metric space, and let $\mathfrak{P}_{cc}(\mathfrak{X})$ denote the collection of all \textit{non-empty compact subsets} of $\mathfrak{X}$. 
    The distance between two points $\mathnormal{x, y} \in \mathfrak{X}$ is given by
    \[
    \mathnormal{d}(\mathnormal{x, y}) = \mathnormal{d}(\mathnormal{y, x}).
    \]
    And between two sets $\mathnormal{A, B} \in \mathfrak{P}_{cc}(\mathfrak{X})$ is given by
    \[
        \text{dist}(\mathnormal{A, B}) := \sup_{\mathnormal{a} \in \mathnormal{A}} \text{dist}(\mathnormal{a, B}) = \sup_{\mathnormal{a} \in \mathnormal{A}} \inf_{\mathnormal{b} \in \mathnormal{B}} \mathnormal{d}(\mathnormal{a, b})
    \]
    
%     Define
%     \[
%         \mathcal{H}(\mathnormal{A, B}) := \max \{ \text{dist}(\mathnormal{A, B}), \text{dist}(\mathnormal{B, A}) \}.
%     \]
    
% This is a metric on $\mathfrak{P}_{cc}(\mathfrak{X})$ called the \textit{Hausdorff metric}.

% \begin{theorem}
%      $(\mathfrak{P}_{cc}(\mathfrak{X}), \mathcal{H})$ is a complete metric space.
% \end{theorem}

\end{frame}


\begin{frame}
    \begin{definition}
        Let $\ell^2$ be the Hilbert space of real-valued square summable bi-infinite sequences
        $\mathnormal{u} = (u_i)_{i \in \mathbb{Z}}$ with norm and inner product
        \[
        \|\mathnormal{u}\| := \left( \sum_{i \in \mathbb{Z}} u_i^2 \right)^{1/2}, \quad \langle \mathnormal{u}, \mathnormal{v} \rangle := \sum_{i \in \mathbb{Z}} u_i v_i \quad \text{for} \quad \mathnormal{u}, \mathnormal{v} \in \ell^2.
        \]
        Note that it is well-known that $\ell^2$ is a complete metric space.
        

    \end{definition}
    
\end{frame}

% \begin{frame}
%     \begin{definition}
%     An autonomous dynamical system on a metric space $(\mathfrak{X}, \mathfrak{d})$ is given by mapping $\varphi : \mathbb{R} \times \mathfrak{X} \to \mathfrak{X}$, which satisfies the properties:
%     \begin{enumerate}[i]
%         \item \textit{initial condition:} $\varphi(0, \mathnormal{x_0}) = \mathnormal{x_0}$ for all $\mathnormal{x_0} \in \mathfrak{X}$,
%         \item \textit{group under composition:} 
%         \[
%         \varphi(\mathnormal{s + t}, \mathnormal{x_0}) = \varphi(\mathnormal{s}, \varphi(\mathnormal{t}, \mathnormal{x_0})) \quad \text{for all } \mathnormal{s, t} \in \mathbb{R}, \, \mathnormal{x_0} \in \mathfrak{X},
%         \]
%         \item \textit{continuity:} the mapping $(\mathnormal{t}, \mathnormal{x}) \mapsto \varphi(\mathnormal{t}, \mathnormal{x})$ is continuous at all points $(\mathnormal{t_0}, \mathnormal{x_0}) \in \mathbb{R} \times \mathfrak{X}$.
%     \end{enumerate}
%     \end{definition}
% \end{frame}

\subsection{Autonomous semi-dynamical systems}
\begin{frame}
    \begin{definition}
        An \textbf{autonomous semi-dynamical system} on a metric space $(\mathfrak{X}, \mathfrak{d})$ is given by a mapping $\varphi \colon \mathbb{R}^+ \times \mathfrak{X} \to \mathfrak{X}$, which satisfies the properties:
        \begin{enumerate}[i]
            \item \textit{initial condition:} 
                $\varphi(0, x_0) = x_0$ for all $x_0 \in \mathfrak{X}$,
                \item \textit{semi-group under composition:} 
                $\varphi(s + t, x_0) = \varphi(s, \varphi(t, x_0))$ for all $s, t \in \mathbb{R}^+$, $x_0 \in \mathfrak{X}$,
                \item \textit{continuity:}
                the mapping $(t, x) \mapsto \varphi(t, x)$ is continuous at all points $(t_0, x_0) \in \mathbb{R}^+ \times \mathfrak{X}$.
        \end{enumerate}
        \end{definition}
\end{frame}



% \begin{frame}
%     \begin{definition}
        
%     An autonomous \textbf{semi-dynamical} dynamical system on a metric space $(\mathcal{X}, d_{\mathcal{X}})$ is given by mapping $\varphi : \mathbb{R}^+ \times \mathcal{X} \rightarrow \mathcal{X}$, which satisfies the properties:

% \begin{itemize}
%     \item[(i)] \textit{initial condition}: $\varphi(0, x_0) = x_0$ for all $x_0 \in \mathcal{X}$,
%     \item[(ii)] \textit{semi-group under composition}:
%     \[
%     \varphi(s + t, x_0) = \varphi(s, \varphi(t, x_0)) \quad \text{for all } s, t \in \mathbb{R}^+, \, x_0 \in \mathcal{X},
%     \]
%     \item[(iii)] \textit{continuity}: the mapping $(t, x) \mapsto \varphi(t, x)$ is continuous at all points $(t_0, x_0) \in \mathbb{R}^+ \times \mathcal{X}$.
% \end{itemize}
% \end{definition}

% \end{frame}

\begin{frame}
    \begin{definition} 
        A semi-dynamical
         system $\{\varphi(t)\}_{t \geq 0}$ on a 
         complete metric space $(\mathfrak{X}, \mathfrak{d})$ is 
         said to be \textbf{asymptotically compact} if, 
         for every sequence $\{t_k\}_{k \in \mathbb{N}}$ 
         in $\mathbb{R}^+$ with $t_k \rightarrow \infty$ 
         as $k \rightarrow \infty$ and every bounded 
         sequence $\{x_k\}_{k \in \mathbb{N}}$ in 
         $\mathfrak{X}$, the sequence 
         $\{\varphi(t_k, x_k)\}_{k \in \mathbb{N}}$ has a 
         convergent subsequence.
    \end{definition}
    \begin{definition}
    A set $B_0 \subset \mathfrak{X}$ is said to be \textbf{absorbing} for a dynamical system $(\mathfrak{X}, \varphi)$ if for any bounded set $B$ in $\mathfrak{X}$ there exists $t_0 = t_0(B)$ such that $\varphi(t, B) \subset B_0$ for every $t \geq t_0$.
    \end{definition}

\end{frame}

\begin{frame}
    A bounded closed set $A_1 \subseteq X$ is called a \textbf{global attractor} for a dynamical system $(\mathfrak{X}, \varphi)$, if
\begin{enumerate}
\item $A_1$ is an invariant set, i.e., $\varphi(t, A_1) = A_1$ for any $t > 0$;

\item the set $A_1$ uniformly attracts all trajectories starting in bounded sets,

i.e., for any bounded set $B$ from $\mathfrak{X}$

\end{enumerate}
$$\limsup_{t \to \infty} \left\{ \mathrm{dist}(\varphi(t, y), A_1) : y \in B \right\} = 0.$$
\end{frame}

\begin{frame}
    \begin{Theorem} Let $\{\varphi(t)\}_{t \geq 0}$ be an autonomous semi-dynamical system on a complete metric space $(\mathfrak{X}, \mathfrak{d})$ which is asymptotically compact and has a closed and bounded absorbing set $Q \subset \mathfrak{X}$. Then $\varphi$ has an attractor $A$, which is contained in $Q$ and is given by
        \[
        A = \bigcap_{t \geq 0} \overline{\bigcup_{s \geq t} \varphi(s, Q)}.
        \]
        
    \end{Theorem}
    
\end{frame}

\begin{frame}
    \textbf{Asymptotic tails property: autonomous systems} 
    Let $\varphi = (\varphi_i)_{i \in \mathbb{Z}}$ be an autonomous semi-dynamical system on the Hilbert space $(\ell^2, || \cdot ||)$ and let $B$ be a positively invariant, closed and bounded subset of $\ell^2$, which is $\varphi$-positive invariant. Then $\varphi$ is said to satisfy an asymptotic tails property in $B$ if for every $\epsilon > 0$ there exist $T(\epsilon) > 0$ and $I(\epsilon) \in \mathbb{N}$ such that 
    
    $$\sum_{|i| > I(\epsilon)} |\varphi_i(t, x_0)|^2 \leq \epsilon \quad \forall x_0 \in B \text{ and } t \geq T(\epsilon).$$

\begin{lemma}[2.5]
    Let Assumption about asymptotic tails property hold.
    Then the semi-dynamical system \(\phi\) is asymptotically compact in \(B\).
\end{lemma}
\end{frame}

% \begin{frame}
%     \begin{definition}
%         A set-valued autonomous dynamical system on a metric space $(\mathfrak{X}, \mathnormal{d}_{\mathfrak{X}})$ is defined in terms of an attainability set mapping $(\mathnormal{t, x}) \mapsto \mathnormal{\Phi}(\mathnormal{t, x})$ on $\mathbb{R}^+ \times \mathfrak{X}$ satisfying
%         \begin{enumerate}[i)]
%             \only<1>{
%             \item \textit{compactness:} $\mathnormal{\Phi}(\mathnormal{t, x_0})$ is a non-empty compact subset of $\mathfrak{X}$ for all $(\mathnormal{t, x_0}) \in \mathbb{R}^+ \times \mathfrak{X}$,
%             \item \textit{initial condition:} $\mathnormal{\Phi}(0, \mathnormal{x_0}) = \{ \mathnormal{x_0} \}$ for all $\mathnormal{x_0} \in \mathfrak{X}$,
%             \item \textit{semi-group:} $\mathnormal{\Phi}(\mathnormal{s + t, x_0}) = \mathnormal{\Phi}(\mathnormal{s}, \mathnormal{\Phi}(\mathnormal{t, x_0}))$ for all $\mathnormal{t, s} \in \mathbb{R}^+$ and all $\mathnormal{x_0} \in \mathfrak{X}$,
%             \item \textit{upper semi-continuity in initial conditions:} $(\mathnormal{t, x}) \mapsto \mathnormal{\Phi}(\mathnormal{t, x})$ is upper semi-continuous in $(\mathnormal{t, x}) \in \mathbb{R}^+ \times \mathfrak{X}$ with respect to the Hausdorff semi-distance $\text{dist}_{\mathfrak{X}}$, i.e.,
%             \[
%             \text{dist}_{\mathfrak{X}}(\mathnormal{\Phi}(\mathnormal{t, x}), \mathnormal{\Phi}(\mathnormal{t_0, x_0})) \to 0 \quad \text{as} \quad (\mathnormal{t, x}) \to (\mathnormal{t_0, x_0}) \quad \text{in} \quad \mathbb{R}^+ \times \mathfrak{X},
%             \]}
%             \only<2>{
%             \setcounter{enumi}{4}
%             \item \textit{continuity:} the mapping $\mathnormal{t} \mapsto \mathnormal{\Phi}(\mathnormal{t, x_0})$ is continuous in $\mathnormal{t} \in \mathbb{R}^+$ with respect to the Hausdorff metric $\mathcal{H}_{\mathfrak{X}}$ uniformly in $\mathnormal{x_0}$ in compact subsets $\mathnormal{B} \in \mathfrak{P}_{cc}(\mathfrak{X})$, i.e.,
%             \[
%             \sup_{\mathnormal{x_0} \in \mathnormal{B}} \mathcal{H}_{\mathfrak{X}}(\mathnormal{\Phi}(\mathnormal{t, x_0}), \mathnormal{\Phi}(\mathnormal{t_0, x_0})) \to 0 \quad \text{as} \quad \mathnormal{t} \to \mathnormal{t_0} \quad \text{in} \quad \mathbb{R}^+.
%             \]}
%         \end{enumerate}
%         \end{definition}        
        
% \end{frame}

% \begin{frame}
%     \begin{definition}
%         A process on a metric space $(\mathfrak{X}, \mathnormal{d}_{\mathfrak{X}})$ is a mapping $\mathnormal{\psi} : \mathbb{R}^2_{\geq 0} \times \mathfrak{X} \to \mathfrak{X}$ with the following properties:
%         \begin{enumerate}[i)]
%             \item \textit{initial condition:} $\mathnormal{\psi}(t_0, t_0, \mathnormal{x_0}) = \mathnormal{x_0}$ for all $\mathnormal{x_0} \in \mathfrak{X}$ and $\mathnormal{t_0} \in \mathbb{R}$.
%             \item \textit{two-parameter semi-group property:}
%             \[
%             \mathnormal{\psi}(t_2, t_0, \mathnormal{x_0}) = \mathnormal{\psi}(t_2, t_1, \mathnormal{\psi}(t_1, t_0, \mathnormal{x_0}))
%             \]
%             for all $(t_1, t_0), (t_2, t_1) \in \mathbb{R}^2_{\geq 0}$ and $\mathnormal{x_0} \in \mathfrak{X}$.
%             \item \textit{continuity:} the mapping $(t, t_0, \mathnormal{x_0}) \mapsto \mathnormal{\psi}(t, t_0, \mathnormal{x_0})$ is continuous.
%         \end{enumerate}
%         \end{definition}
            
% \end{frame}

% \begin{frame}

%     \begin{definition}
%         For an autonomous system, a process reduces to
%         \[
%         \mathnormal{\psi}(t, t_0, \mathnormal{x_0}) = \mathnormal{\varphi}(t - t_0, \mathnormal{x_0}),
%         \]
%         since the solutions depend only on the elapsed time $t - t_0$, i.e., just one parameter instead of independently on the actual time $t$ and the initial time $t_0$, i.e., two parameters.
%     \end{definition}
    
        
% \end{frame}


\section{LDS Problem}
\begin{frame}
    Lattice dynamical systems (LDS) are essentially infinite dimensional systems of ordinary differential equations (ODEs). In particular, they can be formulated as ordinary differential equations on a Hilbert or Banach space of bi-infinite sequences.
    LDS may arise from discretisation of continuum models or as infinite dimensional counterparts of finite ODE models.
    A classical lattice dynamical system is based on a \textbf{reaction-diffusion equation}
\[
\frac{\partial u}{\partial t}=\nu\frac{\partial^{2}u}{\partial x^{2}}-\lambda u+f(u)+g(x),
\]
where $\lambda$ and $\nu$ are positive constants, on a one-dimensional domain $\mathbb{R}$. It is obtained by using a central difference quotient to discretise the Laplacian. Setting the stepsize scaled to equal 1 leads to the infinite dimensional system of ordinary differential equations
\[
\frac{du_{i}}{dt}=\nu(u_{i-1}-2u_{i}+u_{i+1})-\lambda u_{i}+f(u_{i})+g_{i} \quad i\in\mathbb{Z},
\] 
\end{frame}
\begin{frame}
    Consider the autonomous LDS
    \[
    \frac{d\mathnormal{u_i}}{d\mathnormal{t}} = \mathnormal{\nu} (\mathnormal{u_{i-1}} - 2\mathnormal{u_i} + \mathnormal{u_{i+1}}) + \mathnormal{f}(\mathnormal{u_i}) + \mathnormal{g_i}, \quad \mathnormal{i} \in \mathbb{Z}\quad (0)
    \]
    in the space $\ell^2$, which will be investigated here under the following assumptions.
        
\end{frame}

\subsection{Assumptions}
\begin{frame}
    \begin{block}{Assumption 1}
        The function $\mathnormal{f} : \mathbb{R} \to \mathbb{R}$ is a continuously differentiable function satisfying
        \[
        \mathnormal{f(s)}\mathnormal{s} \leq -\alpha \mathnormal{s}^2 \quad \forall \, \mathnormal{s} \in \mathbb{R},
        \]
        for some $\alpha > 0$ with bounded derivative $\mathnormal{f'}$.


        % \textbf{Example:} $\mathnormal{f(s)} = -\mathnormal{s}^3 - \mathnormal{s}$.
    \end{block}
        
    \begin{block}{Assumption 2}
    Let $\boldsymbol{g} = (\mathnormal{g_i})_{i \in \mathbb{Z}} \in \ell^2$.
    \end{block}
\end{frame}


\begin{frame}
    \begin{definition}
        Define the operator $\Lambda : \ell^2 \to \ell^2$ by
        \[
        (\Lambda \mathbf{u})_i = u_{i-1} - 2 u_i + u_{i+1}, \quad i \in \mathbb{Z}
        \]
        and the operators $\mathbf{D}^+, \mathbf{D}^- : \ell^2 \to \ell^2$ by
        \[
        (\mathbf{D}^+ \mathbf{u})_i = u_{i+1} - u_i, \quad (\mathbf{D}^- \mathbf{u})_i = u_{i-1} - u_i, \quad i \in \mathbb{Z}.
        \]
        It is straightforward to check that
        \[
        -\Lambda = \mathbf{D}^+ \mathbf{D}^- = \mathbf{D}^- \mathbf{D}^+ \quad \text{and} \quad \langle \mathbf{D}^- \mathbf{u}, \mathbf{v} \rangle = \langle \mathbf{u}, \mathbf{D}^+ \mathbf{v} \rangle \quad \forall \mathbf{u}, \mathbf{v} \in \ell^2,
        \]
        and hence $\langle \Lambda \mathbf{u}, \mathbf{u} \rangle = - \|\mathbf{D}^+ \mathbf{u}\|^2 \leq 0$ for any $\mathbf{u} \in \ell^2$.
        $\Lambda$ operator is a linear, bounded and nonpositive operator.
    \end{definition}
\end{frame}

\begin{frame}
    The lattice system (0) can be written as an ODE
\[ \frac{du}{dt} = \nu \Lambda u + F(u) + g \quad (*)\]
on $\ell^2$, where $g = (g_i)_{i \in \mathbb{Z}}\in \ell^2 $, $F: \ell^2 \to \ell^2$ is given component wise by $F_i(u) := f(u_i)$ for some continuously differentiable globally Lipschitz function $f: \mathbb{R} \to \mathbb{R}$ with $f(0) = 0$. It follows that the function on the right side of the infinite dimensional ODE \((*)\) maps $\ell^2$ into itself and is globally Lipschitz on $\ell^2$.
\end{frame}

\begin{frame}
\textbf{Theorem 1.2 (Global).} Let the function $f(t, x):\mathbb{R}\times\mathfrak{B}\to\mathfrak{B}$, continuous with respect to $t$, satisfy the following conditions for $t \in [a, b]$, $x \in \mathfrak{B}$:
\begin{align}
\|f(t, x)\| &\leq M_1 + M_0\|x\|,  \\
\|f(t, x_2) - f(t, x_1)\| &\leq M_2\|x_2 - x_1\|, 
\end{align}
where $M_0$, $M_1$, and $M_2$ are constants.
Then, for any $x_0 \in \mathfrak{B}$ and $t_0 \in [a, b]$, the differential equation $\frac{dx}{dt} = f(t, x)$ has a unique solution $x = \varphi(t)$ on the entire interval $[a, b]$, satisfying the initial condition $\varphi(t_0)=x_0$.

\end{frame}

\begin{frame}
Under Theorem 1.2 conditions, equation \((*)\) generates a semigroup dynamical system $(\ell^{2},\varphi)$ in the space $\ell^{2}$, where $\varphi(t,x):=u(t,x)$ and $u(t,x)$ is the solution of equation \((*)\) with the initial condition $u(0,x)=x$ ($x\in \ell^{2}$).
\end{frame}

% \begin{frame}
%     For any $\boldsymbol{u} = (\mathnormal{u_i})_{i \in \mathbb{Z}} \in \ell^2$, the discrete Laplace operator $\Lambda$ is defined from $\ell^2$ to $\ell^2$ component-wise by
% \[
% (\Lambda \boldsymbol{u})_i = \mathnormal{u_{i-1}} - 2\mathnormal{u_i} + \mathnormal{u_{i+1}}, \quad \mathnormal{i} \in \mathbb{Z}.
% \]
% \end{frame}

\subsection{Existence of a global attractor}
\begin{frame}{LDS Problem}
   \begin{theorem}
       Suppose that Assumptions are satisfied. The autonomous semi-dynamical system $\{\mathnormal{\varphi}(t)\}_{t \geq 0}$ generated by the ODE \((*)\) on $\ell^2$ has a global attractor $\mathcal{A}$ in $\ell^2$.
   \end{theorem}
   \[ \frac{du}{dt} = \nu \Lambda u + F(u) + g \quad (*)\]
\end{frame}

\section{Proof of the main result}
\subsection{ Existence of an absorbing set}
\begin{frame}
    Taking the inner product of equation \((*)\) with $\boldsymbol{u}$ gives
    \[
    \frac{d}{dt} \|\boldsymbol{u}\|^2 = 2\mathnormal{\nu} \langle \Lambda \boldsymbol{u}, \boldsymbol{u} \rangle + 2 \langle \mathnormal{F}(\boldsymbol{u}), \boldsymbol{u} \rangle + 2 \langle \boldsymbol{g}, \boldsymbol{u} \rangle
    \]
    \[
    \leq 2 \sum_{i \in \mathbb{Z}} \mathnormal{u_i} \mathnormal{f(u_i)} + 2 \sum_{i \in \mathbb{Z}} \mathnormal{g_i} \mathnormal{u_i} \leq -\alpha \|\boldsymbol{u}\|^2 + \frac{\|\boldsymbol{g}\|^2}{\alpha},
    \]
    Since Assumption 1 gives \(2\sum_{i\in\mathbb{Z}}\mathnormal{u_i} f(u_i) \leq -2\alpha\|\boldsymbol{u}\|^2\)
    and Young's inequality gives \(2\sum_{i\in\mathbb{Z}}\mathnormal{g_i u_i}\leq\alpha\|\boldsymbol{u}\|^2+\|\boldsymbol{g}\|^2/\alpha\). Hence, Gronwall's lemma implies that
    \[
    \|\boldsymbol{u}(t)\|^2 \leq \|\boldsymbol{u}_0\|^2 e^{-\alpha t} + \frac{\|\boldsymbol{g}\|^2}{\alpha^2} \left(1 - e^{-\alpha t}\right), \quad t \geq 0 \quad (3.7)
    \]
    
    Define the closed ball $Q$ in $\ell^2$ by
    \[
    Q := \left\{ \boldsymbol{u} \in \ell^2 : \|\boldsymbol{u}\|^2 \leq R^2 := 1 + \frac{\|\boldsymbol{g}\|^2}{\alpha^2} \right\}.
    \]
    
\end{frame}

\begin{frame}
    The estimate then implies that $Q$ is an absorbing set for the autonomous semi-dynamical system $\varphi$. In fact, for any $\boldsymbol{u_0} \in B$, which is a bounded set in $\ell^2$, it is straightforward to check that
\[
\varphi(t, B) \subset Q \quad \forall t \geq \frac{2}{\alpha} \ln \|B\|,
\]
where $\|B\| := \sup_{\boldsymbol{u} \in B} \|\boldsymbol{u}\|$.

Moreover, for every $\boldsymbol{u_0} \in Q$, by estimate (3.7) we have
\[
\|\varphi(t, \boldsymbol{u_0})\|^2 \leq R^2 e^{-\alpha t} + R^2 (1 - e^{-\alpha t}) = R^2,
\]
 
\end{frame}

\subsection{Asymptotic tails property}
\begin{frame}
    In order to establish the asymptotic compactness property in Assumption 2 for the autonomous semi-dynamical system $\{\varphi(t)\}_{t \geq 0}$ in $\ell^2$, the asymptotic tails property needs to be shown to hold.
    Consider a smooth function $\xi : \mathbb{R}^+ \to [0, 1]$ satisfying
    \[
    \xi(s) =
    \begin{cases}
    0, & 0 \leq s \leq 1, \\
    \in [0, 1], & 1 \leq s \leq 2, \\
    1, & s \geq 2,
    \end{cases}
    \]
    and note that there exists a constant $C_0$ such that $|\xi'(s)| \leq C_0$ for all $s \geq 0$. Notice that this function $\xi$ or a similar function will be used repeatedly throughout this book. Then for a fixed $k \in \mathbb{N}$ (its value will be specified later), define
    \[
    \xi_k(s) = \xi\left(\frac{s}{k}\right) \quad \text{for all} \quad s \in \mathbb{R}.
    \]
  
\end{frame}

\begin{frame}
        
    Given $\boldsymbol{u} \in \ell^2$, define $\boldsymbol{v} \in \ell^2$ component-wise as
    \[
    v_i := \xi_k(|i|) u_i \quad i \in \mathbb{Z}.
    \]
    
    Taking the inner product of equation \((*)\) with $\boldsymbol{v}$ gives
    \[
    \frac{d}{dt} \langle \boldsymbol{u}, \boldsymbol{v} \rangle + \nu \langle \mathnormal{D}^+ \boldsymbol{u}, \mathnormal{D}^+ \boldsymbol{v} \rangle = \langle \mathnormal{F}(\boldsymbol{u}), \boldsymbol{v} \rangle + \langle \boldsymbol{g}, \boldsymbol{v} \rangle.
    \]
    that is
    \[
    \frac{d}{dt} \sum_{i \in \mathbb{Z}} \xi_k(|i|) |u_i|^2 + 2\mathnormal{\nu} \langle \mathnormal{D}^+ \boldsymbol{u}, \mathnormal{D}^+ \boldsymbol{v} \rangle 
    = 2 \sum_{i \in \mathbb{Z}} \xi_k(|i|) u_i \mathnormal{f}(u_i) + 2 \sum_{i \in \mathbb{Z}} \xi_k(|i|) g_i u_i

    \quad (3.9)
    \]

\end{frame}

\begin{frame}
        
    Each term in equation will now be estimated. First,
    \[
    \langle \mathnormal{D}^+ \boldsymbol{u}, \mathnormal{D}^+ \boldsymbol{v} \rangle = \sum_{i \in \mathbb{Z}} (u_{i+1} - u_i)(v_{i+1} - v_i)
    \]
    \[
    = \sum_{i \in \mathbb{Z}} (u_{i+1} - u_i) \left[ (\xi_k(|i+1|) - \xi_k(|i|)) u_{i+1} + \xi_k(|i|)(u_{i+1} - u_i) \right]
    \]
    \[
    = \sum_{i \in \mathbb{Z}} (\xi_k(|i+1|) - \xi_k(|i|)) (u_{i+1} - u_i) u_{i+1} + \sum_{i \in \mathbb{Z}} \xi_k(|i|)(u_{i+1} - u_i)^2
    \]
    \[
    \geq \sum_{i \in \mathbb{Z}} (\xi_k(|i+1|) - \xi_k(|i|)) (u_{i+1} - u_i) u_{i+1}.
    \]
    
\end{frame}

\begin{frame}
    Since
\[
\sum_{i \in \mathbb{Z}} \left( \xi_k(|i+1|) - \xi_k(|i|) \right) (u_{i+1} - u_i) u_{i+1} \leq \sum_{i \in \mathbb{Z}} \frac{1}{k} |\xi'(s_i)| \cdot |u_{i+1} - u_i| \cdot |u_i| + |u_{i+1}|,
\]
for some $s_i$ between $|i|$ and $|i+1|$, and
\[
\sum_{i \in \mathbb{Z}} |\xi'(s_i)| |u_{i+1} - u_i| |u_{i+1}| \leq C_0 \sum_{i \in \mathbb{Z}} \left( |u_{i+1}|^2 + |u_i| |u_{i+1}| \right) \leq 4C_0 \|\boldsymbol{u}\|^2.
\]

Then it follows that for all $\boldsymbol{u} \in Q$ and $\boldsymbol{v} \in \ell^2$ defined component-wise as $v_i := \xi_k(|i|) u_i$ for $i \in \mathbb{Z}$,
\[
\langle \mathnormal{D}^+ \boldsymbol{u}, \mathnormal{D}^+ \boldsymbol{v} \rangle \geq - \frac{4C_0 \|Q\|^2}{k}  \quad (3.10)
\]
where $\|Q\| := \sup_{\boldsymbol{u} \in Q} \|\boldsymbol{u}\|$.

\end{frame}

\begin{frame}
    On the other hand, by Assumption 1,
\[
2 \sum_{i \in \mathbb{Z}} \xi_k(|i|) u_i f(u_i) \leq -2 \alpha \sum_{i \in \mathbb{Z}} \xi_k(|i|) |u_i|^2
\]
and by Young's inequality
\[
2 \sum_{i \in \mathbb{Z}} \xi_k(|i|) g_i u_i \leq \alpha \sum_{i \in \mathbb{Z}} \xi_k(|i|) |u_i|^2 + \frac{1}{\alpha} \sum_{i \in \mathbb{Z}} \xi_k(|i|) |g_i|^2.
\]
Thus,
\[
2 \sum_{i \in \mathbb{Z}} \xi_k(|i|) u_i f(u_i) + 2 \sum_{i \in \mathbb{Z}} \xi_k(|i|) g_i u_i 

\leq -\alpha \sum_{i \in \mathbb{Z}} \xi_k(|i|) |u_i|^2 + \frac{1}{\alpha} \sum_{|i| \geq k} |g_i|^2 \quad (3.11).
\]

Using the estimates (3.10) and (3.11) in equation (3.9) gives
\[
\frac{d}{dt} \sum_{i \in \mathbb{Z}} \xi_k(|i|) |u_i|^2 + \alpha \sum_{i \in \mathbb{Z}} \xi_k(|i|) |u_i|^2 \leq \nu \frac{4C_0 \|Q\|^2}{k} + \frac{1}{\alpha} \sum_{|i| \geq k} |g_i|^2 \quad (3.12).
\]

\end{frame}

\begin{frame}
    The inequality (3.12) along with the relation above give
\[
\frac{d}{dt} \sum_{i \in \mathbb{Z}} \xi_k(|i|) |u_i|^2 + \alpha \sum_{i \in \mathbb{Z}} \xi_k(|i|) |u_i|^2 \leq \epsilon.
\]

Then, Gronwall's lemma implies that
\[
\sum_{i \in \mathbb{Z}} \xi_k(|i|) |u_i(t, \boldsymbol{u_0})|^2 \leq e^{-\alpha t} \sum_{i \in \mathbb{Z}} \xi_k(|i|) |u_{0,i}|^2 + \frac{\epsilon}{\alpha} \leq e^{-\alpha t} \|\boldsymbol{u_0}\|^2 + \frac{\epsilon}{\alpha}.
\]

Hence for every $\boldsymbol{u_0} \in Q$,
\[
\sum_{i \in \mathbb{Z}} \xi_k(|i|) |u_i(t, \boldsymbol{u_0})|^2 \leq e^{-\alpha t} \|Q\|^2 + \frac{\epsilon}{\alpha},
\]

and therefore
\[
\sum_{i \in \mathbb{Z}} \xi_k(|i|) |u_i(t, \boldsymbol{u_0})|^2 \leq \frac{2 \epsilon}{\alpha}, \quad \text{for} \quad t \geq T(\epsilon) := \frac{1}{\alpha} \ln \frac{\alpha \|Q\|^2}{\epsilon}.
\]

\end{frame}

\begin{frame}
    This is the desired asymptotic tails property. The asymptotic compactness in $\ell^2$ of $\varphi$ in the absorbing set $Q$ then follows from Lemma 2.5. This completes the proof of the Theorem.

\end{frame}
\end{document}
