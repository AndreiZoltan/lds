%%%%%% Latex file
%%                              Almost Periodic Solutions of Lattice Dynamical Systems with Monotone Nonlinearity
%%                                             David Cheban and Andrei Sultan
%%                                                    June, 2025
%%%%%%%%%%%%%%%%%%%%%%%%%%%%%%


%\usepackage{showkeys}
%\usepackage{amscd}
%\usepackage{latexsym}
%\usepackage{oldlfont}
%\usepackage{latexsym}
%\usepackage{oldlfont}
%\renewcommand{\baselinestretch}{1.5}-
%\input amssym.def
%\input amssym
%% Title
%\title[]{%
% First author
% Second author

\documentclass{amsart}%
\usepackage{amsmath}
\usepackage{amsfonts}
\usepackage{amssymb}
\usepackage{graphicx}%
\usepackage{xcolor}
\usepackage{hyperref}
\setcounter{MaxMatrixCols}{30}
%TCIDATA{OutputFilter=latex2.dll}
%TCIDATA{Version=5.00.0.2552}
%TCIDATA{LastRevised=Wednesday, January 09, 2008 22:34:42}
%TCIDATA{<META NAME="GraphicsSave" CONTENT="32">}
%TCIDATA{<META NAME="SaveForMode" CONTENT="1">}
%TCIDATA{Language=American English}
\def\eps{\varepsilon}
\newtheorem{lemma}{Lemma}[section]
\newtheorem{theorem}[lemma]{Theorem}
\newtheorem{remark}[lemma]{Remark}
\newtheorem{prop}[lemma]{Proposition}
\newtheorem{coro}[lemma]{Corollary}
\newtheorem{definition}[lemma]{Definition}
\newtheorem{example}[lemma]{Example}
\renewcommand{\labelenumi}{(\roman{enumi})}
\parindent0.0em
\parskip0.7em
%\email[T.~Caraballo]{caraball@us.es}
%\email[D.Cheban]{cheban@usm.md}

\title[Almost Periodic Solutions of Lattice Dynamical Systems \dots ]{Almost Periodic Solutions of Lattice Dynamical Systems with Monotone Nonlinearity}

% First author
\author{David Cheban}
\address[D. Cheban]{State University of Moldova\\
Vladimir Andrunachievici Instiy=tute of Mathematics and Computer Science\\
Laboratory of Differential Equations\\
A. Mateevich Street 60\\
MD--2009 Chi\c{s}in\u{a}u, Moldova} \email[D.
Cheban]{david.ceban@usm.md, davidcheban@yahoo.com}
% Second author
\author{Andrei Sultan}
\address[A. Sultan]{%
State University of Moldova\\Vladimir Andrunachievici Institute of
Mathematics and Computer Science\\ Laboratory of Differential Equations\\A. Mateevich Street 60\\
MD--2009 Chi\c{s}in\u{a}u, Moldova} \email[A.
Sultan]{andrew15sultan@gmail.com}





\date{\today}
\subjclass{34D05, 34D45, 34G20, 37B55} \keywords{Lattice Dynamical
Systems; Monotone non-autonomous Dynamical Systems; Almost
Periodic Solutions}

%%%%%%%%%%%%%%%%%%%%%%%%%%%%%%
%%%%%%%%%%%%%%%%%%%%%%%%%%%%%%
%%%%%%%%%%


\begin{document}

\begin{abstract}
The aim of this paper is studying the problem of almost
periodicity of almost periodic lattice dynamical systems of the
form $u_{i}'=\nu (u_{i-1}-2u_i+u_{i+1})-\lambda
u_{i}+F(u_i)+f_{i}(t)\ (i\in \mathbb Z,\ \lambda >0)$. We prove
the existence a unique almost periodic solution of this system if
the nonlinearity $F$ is monotone.
\end{abstract}

\maketitle



\section{Introduction}\label{Sec1}

Denote by $\mathbb R :=(-\infty,\infty)$, $\mathbb Z :=\{0,\pm
1,\pm 2,\ldots\}$ and $\ell_{2}$ the Hilbert space of all
two-sided sequences $\xi =(\xi_{i})_{i\in \mathbb Z}$ ($\xi_{i}\in
\mathbb R$) with
\begin{equation}\label{eqI_1}
\sum\limits_{i\in \mathbb Z}|\xi_{i}|^{2}<+\infty \nonumber
\end{equation}
and equipped with the scalar product
\begin{equation}\label{eqI_2}
\langle \xi,\eta\rangle :=\sum\limits_{i\in \mathbb
Z}\xi_{i}\eta_{i} .\nonumber
\end{equation}

Let $(X,\rho)$ be a complete metric space with the distance
$\rho$, $C(\mathbb R,X)$ be the space of all continuous functions
$f:\mathbb R\to X$ equipped with the distance
\begin{equation}\label{eqI_3}
d(f_1,f_2):=\sup\limits_{L>0}\min\{\max\limits_{|t|\le
L}\rho(f_{1}(t),f_{2}(t)),L^{-1}\}.
\end{equation}
The metric space $(C(\mathbb R,X),d)$ is complete and the distance
$d$, defined by (\ref{eqI_3}), generates on the space $C(\mathbb
R,X)$ the compact-open topology.

Let $h\in \mathbb R$, $f\in C(\mathbb R,X)$, $f^{h}(t):=f(t+h)$
for any $t\in \mathbb R$ and $\sigma :\mathbb R\times C(\mathbb
R,X)\to C(\mathbb R,X)$ be a mapping defined by
$\sigma(h,f):=f^{h}$ for any $(h,f)\in \mathbb R\times C(\mathbb
R,X)$. Then \cite[Ch.I]{Che_2015} the triplet $(C(\mathbb
R,X),\mathbb R,\sigma)$ is a shift dynamical system (or Bebutov's
dynamical system) on he space $C(\mathbb R,X)$. By $H(f)$ the
closure in the space $C(\mathbb R,X)$ of $\{f^{h}|\ h\in \mathbb
R\}$ is denoted.

Recall that a subset $A\subset \mathbb R$ is called relatively
dense if there exits a positive number $l$ such that
\begin{equation}\label{eqRD1}
A\bigcap [a,a+l]\not= \emptyset
\end{equation}
for any $a\in \mathbb R$.

A function $f\in C(\mathbb R,X)$ is called almost periodic
\cite{Che_2020,Lev-Zhi}, if for any positive number $\varepsilon$
the set
\begin{equation}\label{eqRD2}
\mathcal F(f,\varepsilon):=\{\tau \in \mathbb R |\
\rho(f(t+\tau),f(t))<\varepsilon\ \ \mbox{for any}\ t\in \mathbb
R\}
\end{equation}
is relatively dense.

In this paper we study the problem of existence at least one
almost periodic solution of the systems
\begin{equation}\label{eqI1}
u_{i}'=\nu (u_{i-1}-2u_i+u_{i+1})-\lambda u_{i}+F(u_i)+f_{i}(t)\
(i\in \mathbb Z),
\end{equation}
where $\lambda >0$, $F\in C(\mathbb R, \mathbb R)$ and $f\in
C(\mathbb R,\ell_{2})$ ($f(t):=(f_{i}(t))_{i\in \mathbb Z}$ for
any $t\in \mathbb R$) is an almost periodic function.

The system (\ref{eqI1}) can be considered as a discrete (see, for
example, \cite{BLW_2001,HK_2023} and the bibliography therein)
analogue of a reaction-diffusion equation in $\mathbb R$:
\begin{equation}\label{eqI1.1}
 \frac{\partial{u}}{\partial{t}} = D\frac{\partial^{2}{u}}{\partial^{2}{x}}-\lambda u + F(u) +
 f(t,x),\nonumber
\end{equation}
where grid points are spaced $h$ distance apart and $\nu =
D/h^{2}$.

This study continues the authors works \cite{CS_2025} devoted to
the study the problem of existence of compact global attractor for
(\ref{eqI1}) and the work \cite{CS} dedicated to the study the
invariant sections of monotone nonautonomous dynamical systems and
their applications to differen classes of evolution equations. The
invariant sections play a very important role in the study the
problem of existence of almost periodic (respectively, almost
automorphic, recurrent and Poisson stable) solutions of
differential equations.

The paper is organized as follows. In the second section we show
that under some conditions the equation (\ref{eqI1}) generates a
cocycle which plays a very important role in the study of the
asymptotic properties of the equation (\ref{eqI1}). In the third
section we prove that under some conditions the existence of a
compact global attractor for the equation (\ref{eqI1}). The fourth
section is dedicated to the study the invariant sections of the
cocycle generated by the equation (\ref{eqI1}). In the fifth
section we study the structure of the compact global attractor for
the equation (\ref{eqI1}). Namely, we show that the equation
(\ref{eqI1}) is convergent, i.e., it admits a compact global
attractor $\mathcal I =\{I_{y}|\ y\in Y\}$ such that every set
$I_{y}$ consists of a single point. Finally, we are analyzing an
example of the equation of the form (\ref{eqI1}) which illustrate
our general results.


\section{Cocycles}\label{Sec2}

Consider a non-autonomous system
\begin{equation}\label{eq2.1}
u_{i}'=\nu (u_{i-1}-2u_i+u_{i+1})-\lambda u_{i}+F(u_i)+f_{i}(t)\
(i\in \mathbb Z) .
\end{equation}

Below we use the following conditions.

\emph{Condition }(\textbf{C1}). \label{C1} The function $f\in
C(\mathbb R,\mathfrak B)$ is almost periodic.

\emph{Condition} (\textbf{C2}). \label{C2} The function $F\in
C(\mathbb R,\mathbb R)$ is Lipschitz continuous on bounded sets
and $F(0)=0$.

Denote by $ \widetilde{F}:\ell_{2}\to \ell_{2}$ the Nemytskii
operator generated by $F$, i.e.,
$\widetilde{F}(\xi)_{i}:=F(\xi_{i})$ for any $i\in \mathbb Z$.

\emph{Condition} (\textbf{C3}). \label{C3} The function $F$ is
monotone, i.e., there exists a number $\alpha \ge 0$ such that
\begin{equation}\label{eqAP2}
(x_1-x_2)(F(x_1)-F(x_2))\leq -\alpha
|x_1-x_2|^2
\end{equation}
for any $x_1,x_2 \in \mathbb R$.

\begin{lemma}\label{lAP1} The following statements hold:
\begin{enumerate}
\item if the function $f$ satisfies the Conditions (C1), (C3) and
$F(0)=0$, then
\begin{equation}\label{eqAP3}
F(s)s\le -\alpha s^{2}
\end{equation}
for any $s\in \mathbb R$; \item if the function $F$ satisfies the
Condition (C3), then the Nemytskii operator $\widetilde{F}$
generated by $F$ possesses the following property
\begin{equation}\label{eqAP4}
\langle u_1-u_2,\widetilde{F}(u_1)-\widetilde{F}(u_2)\rangle \le
-\alpha \|u_1-u_2\|^{2}
\end{equation}
for any $u_1,u_2\in \ell_{2}$.
\end{enumerate}
\end{lemma}
\begin{proof}

\end{proof}


\begin{definition}\label{defL1.8} A function $F\in C(Y\times \mathfrak B,\mathfrak
B)$ is said to be Lipschitzian on bounded subsets from $\mathfrak
B$ if for any bounded subset $B\subset \mathfrak B$ there exists a
positive constant $L_{B}$ such that
\begin{equation}\label{eqL1.82}
|F(y,v_1)-F(y,v_2)|\le L_{B} |v_1-v_2|
\end{equation}
for any $v_1, v_2 \in B \subset \mathfrak B$.
\end{definition}

\begin{definition}\label{defL2.8}
The smallest constant $L$ (respectively $L_{B}$) with the property
(\ref{eqL1.81}) is called Lipshchitz constant of function $F$
(notation $Lip(F)$ (respectively, $Lip_{B}(F)$)).
\end{definition}

Let $B \subset \mathfrak B$, denote by $CL(Y\times B,\mathfrak B)$
the Banach space of any Lipschitzian functions $F\in C(Y\times
B,\mathfrak B)$ equipped with the norm
\begin{equation}
||F||_{CL}:=\max\limits_{y\in Y}|F(y,0)|+Lip_{B}(F).\nonumber
\end{equation}

\begin{lemma}\label{l2.2} \cite{BLW_2001} Under the Condition (\textbf{C2})
it is well defined the mapping $\widetilde{F}:\ell_{2}\to
\ell_{2}$ and
\begin{equation}\label{eq2.2}
\|\widetilde{F}(\xi)-\widetilde{F}(\eta)\|\le Lip_{B}(F)\|\xi
-\eta \| \nonumber
\end{equation}
for any $\xi,\eta\in \ell_{2}$, where $\|\cdot\|^{2}:=\langle
\cdot,\cdot \rangle$ and $\|\cdot \|$ is the norm on the space
$\ell_{2}$.
\end{lemma}

For any $u = (u_{i})_{i\in \mathbb Z}$, the discrete Laplace
operator $\Lambda$ is defined \cite[Ch.III]{HK_2023} from
$\ell_{2}$ to $\ell_{2}$ component wise by $\Lambda(u)_{i} =
u_{i-1} - 2u_{i} + u_{i+1}$ ($i\in \mathbb Z$). Define the
bounded linear operators $D^{+}$ and $D^{-}$ from $\ell_{2}$ to
$\ell_{2}$ by $(D^{+}u)_{i} = u_{i+1} - u_{i},\ (D^{-}u)_{i} =
u_{i-1} - u_{i}\ (i\in \mathbb Z)$.

Note that $\Lambda = D^{+}D^{-} = D^{-}D^{+}$ and $\langle D^{-}u,
v\rangle = \langle u, D^{+}v\rangle $ for any $u,v\in \ell_{2}$
and, consequently, $\langle \Lambda u,u \rangle = -|D^{+}u|^{2}\le
0$. Since $\Lambda$ is a bounded linear operator acting on the
space $\ell_{2}$, it generates a uniformly continuous semi-group
on $\ell_{2}$.

Under the Conditions (\textbf{C1}) and (\textbf{C2}) the system of
differential equations (\ref{eq2.1}) can be written in the form of
an ordinary differential equation
\begin{equation}\label{eq2.3}
u'=\nu \Lambda u +\Phi (u)+f(t)
\end{equation}
in the Banach space $\mathfrak B=\ell_{2}$, where
$\Phi(u):=-\lambda u +\widetilde{F} (u)$ and
$\Lambda(u)_{i}:=u_{i-1}-2u_{i}+u_{i+1}$ for any $u=(u_i)_{i\in
\mathbb Z}\in \ell_{2}$. Along with the equation (\ref{eq2.3}) we
consider also it $H$-class, i.e., the family equations
\begin{equation}\label{eq2.3g}
u'=\nu \Lambda u +\Phi (u)+g(t),
\end{equation}
where $g\in H(f)$.

The family of the equations (\ref{eq2.3g}) can be rewritten as
follows
\begin{equation}\label{eq2.3H}
u'=F(\sigma(t,g),u)\ \ (g\in H(f)),
\end{equation}
where $F:H(f)\times \ell_{2}\to \ell_{2}$ is defined by
$F(g,u):=\nu \Lambda u+\Phi (u) +g(0)$. It easy to see that
$F(\sigma(t,g),u)=\nu \Lambda u+\Phi (u)+g(t)$ for any $(t,u,g)\in
\mathbb R\times \mathfrak B \times H(f)$.

Let $(Y,\mathbb R,\sigma)$ be a dynamical system on the metric
space $Y$.

\begin{theorem}\label{thAP1} \cite{CS_2025} Under the Conditions (\textbf{C1})-(\textbf{C3}) the following statements hold:
\begin{enumerate}
    \item for any $(v,g)\in \ell_{2}\times H(f)$ there exists a unique
    solution $\varphi(t,v,g)$ of the equation (\ref{eq2.3g}) passing
    through the point $v$ at the initial moment $t=0$ and defined on
    the semi-axis $\mathbb R_{+}:=[0,+\infty)$; \item
    $\varphi(0,v,g)=v$ for any $(v,g)\in \ell_{2}\times H(f)$; \item
    $\varphi(t+\tau,v,g)=\varphi(t,\varphi(\tau,v,g),g^{\tau})$ for
    any $t,\tau\in \mathbb R_{+}$, $v\in \ell_{2}$ and $g\in H(f)$;
     \item the mapping
    $\varphi :\mathbb R_{+}\times \ell_{2}\times H(f)\to \ell_{2}$
    ($(t,v,g)\to \varphi(t,v,g))$ for any $(t,v,g)\in \mathbb
    R_{+}\times \ell_{2}\times H(f)$ is continuous.
\end{enumerate}
\end{theorem}

Let $Y$ be a complete metric space and $(Y,\mathbb R,\sigma)$ be a
dynamical system on $Y$.

\begin{definition}\label{defC1} Recall \cite[Ch.I]{Che_2015} that
$\langle \mathfrak B,\varphi, (Y,\mathbb R,\sigma)\rangle$ is said
to be a cocycle over $(Y,\mathbb R,\sigma)$ with the fiber
$\mathfrak B$ if $\varphi$ is a continuous mapping acting from
$\mathbb R_{+}\times \mathfrak B\times Y$ to $\mathfrak B$ and
satisfying the following conditions:
\begin{enumerate}
\item $\varphi(0,u,y)=v$ for any $(v,y)\in \mathfrak B\times Y$;
\item
$\varphi(t+\tau,u,y)=\varphi(t,\varphi(\tau,u,t),\sigma(\tau,y))$
for any $t,\tau \in \mathbb R_{+}$ and $(u,y)\in \mathfrak B\times
Y$.
\end{enumerate}
\end{definition}

\begin{coro}\label{corH1}
Under the conditions of Theorem \ref{thAP1} the equation
(\ref{eq2.3}) (respectively, the family of equations
(\ref{eq2.3g})) generates a cocycle $\langle
\ell_{2},\varphi,(H(f),\mathbb R,\sigma)\rangle$ over the shift
dynamical system $(H(f),\mathbb R,\sigma)$ with the fiber
$\ell_{2}$.
\end{coro}

\begin{theorem}\label{thAP2} Under the conditions (C1)-(C3) the
cocycle $\langle \ell_{2},\varphi, (H(f),\mathbb R,\sigma)\rangle$
generated by the equation (\ref{eq2.3}) possesses the following
property:
\begin{equation}\label{eqAP4}
\|\varphi(t,v_1,g)-\varphi(t,v_2,g)\|\le e^{-(\lambda +\alpha)
t}\|v_1-v_2\|
\end{equation}
for any $v_1,v_2\in \ell_{2}$, $t\ge 0$ and $g\in H(f)$.
\end{theorem}
\begin{proof}

\end{proof}

\section{Compact global attractors}

\begin{definition}\label{defCGA0_1} A family $\{I_{y}|\ y\in Y\}$ of
compact subsets $I_{y}$ of $\mathfrak B$ is said to be a compact
global attractor for the cocycle $\langle \mathfrak
B,\varphi,(Y,\mathbb R,\sigma)\rangle$ if the following conditions
are fulfilled:
\begin{enumerate}
\item the set
\begin{equation}\label{eqCGA1}
\mathcal I :=\bigcup \{I_{y}|\ y\in Y\}\nonumber
\end{equation}
is precompact; \item the family of subsets $\{I_{y}|\ y\in Y\}$ is
invariant, i.e., $\varphi(t,I_{y},y)=I_{\sigma(t,y)}$ for any
$(t,y)\in \mathbb R_{+}\times Y$; \item
\begin{equation}\label{eqCGA2}
\lim\limits_{t\to +\infty}\sup\limits_{y\in
Y}\beta(\varphi(t,M,y),\mathcal I)=0\nonumber
\end{equation}
\end{enumerate}
for any compact subset $M$ from $\mathfrak B$.
\end{definition}


\begin{definition}\label{defCGA1} A cocycle $\varphi$ is said to be
dissipative if there exists a bounded subset $K\subset \mathfrak
B$ such that for any bounded subset $B\subset \ \mathfrak B$ there
exists a positive number $L=L(B)$ such that
$\varphi(t,B,Y)\subseteq K$ for any $t\ge L(B)$, where
$\varphi(t,B,Y):=\{\varphi(t,u,y)|\ (u,y)\in B\times Y\}$.
\end{definition}


\begin{theorem}\label{thCGA1} \cite[Ch.II]{Che_2024} Assume that the metric space $Y$ is
compact and the cocycle $\langle \mathfrak B,\varphi,(Y,\mathbb
R,\sigma)\rangle$ is dissipative and asymptotically compact.

Then the cocycle $\varphi$ has a compact global attractor.
\end{theorem}

\begin{theorem}\label{thCGA2} Under the Conditions
(\textbf{C1})-(\textbf{C3}) the equation (\ref{eq2.3}) (the
cocycle $\varphi$ generated by the equation (\ref{eq2.3})) has a
compact global attractor $\{I_{g}|\ g\in H(f)\}$.
\end{theorem}
\begin{proof}

\end{proof}


\section{Invariant sections of monotone nonautonomous dynamical
systems}


Below we prove that under some conditions a nonautonomous
dynamical system admits an invariant continuous section.

Let $(Y,\mathbb{R},\sigma)$ be a two-sided dynamical system,
$(X,\mathbb{R_{+}}, \pi)$ be a semi-group dynamical system and
$h:X\to Y$ be a homomorphism of $(X,\mathbb{R_{+}},\pi)$ onto
$(Y,\mathbb{R},\sigma)$.

Let $\langle(X,\mathbb{R_{+}},\pi),(Y,\mathbb{R},\sigma),h\rangle$
be a nonautonomous dynamical system.

Recall that a point $x\in X$ (respectively, a motion $\pi(t,x)$) is
said to be almost periodic, if the continuous mapping
$\pi(\cdot,x):\mathbb T\to X$ is almost periodic.

\begin{definition} A mapping $\gamma :Y\mapsto X$ is called a
continuous invariant section if the following conditions are
fulfilled:
\begin{enumerate}
\item $h(\gamma(y))=y$ for all $y\in Y$; \item
$\gamma(\sigma(t,y))=\pi(t,\gamma(y))$ for any $y\in Y$ and $t\in
R_{+}$; \item $\gamma$ is continuous.
\end{enumerate}
\end{definition}

\begin{lemma}\label{lCIS1} Let $(X,\mathbb T_{1},\pi)$ and $(Y,\mathbb T_{2},\sigma)$ be two dynamical systems,
$\mathbb T_{1}\subseteq \mathbb T_{2}$ and $\gamma :Y\mapsto X$ be
a continuous invariant section. If the point $x\in X$
(respectively, the motion $\pi(t,x)$) is almost periodic, then the
point $y=h(x)$ (respectively, the motion
$\sigma(y,t)=\sigma(t,h(x))=h(\pi(t,x))$ for any $t\in \mathbb
T_{1}$) is also almost periodic.
\end{lemma}
\begin{proof}

\end{proof}


\begin{remark}\label{r3.1}
A continuous section $\gamma\in\Gamma(Y,X)$ is invariant if and
only if $\gamma\in\Gamma(Y,X)$ is a stationary point of the
semigroup $\{S^t\ \vert \ t\in \mathbb{R}_{+} \}$, where $S^t:
\Gamma(Y,X)\to \Gamma(Y,X)$ is defined by the equality
$(S^t\gamma)(y):=\pi(t,\gamma(\sigma(-t,y)))$ for all $y\in Y$ and
$t\in \mathbb R_+$.
\end{remark}


\begin{theorem}\label{thAPM1} Let $\langle \ell_{2} ,\varphi, (H(f),\mathbb
R,\sigma)\rangle$be the cocycle generated by the equation
(\ref{eq2.3}). Under the condition (C1)-(C3) there exists a unique
invariant section $\nu :H(f)\to \ell_{2}$ of the cocycle $\varphi$
and
\begin{equation}\label{eqAPM1}
\|\varphi(t,v,g)-\nu(\sigma(t,g))\|\le e^{-\alpha t}\|v-\nu(g)\|
\end{equation}
for any $t\in \mathbb R_{+}$ and $v\in \ell_{2}$.
\end{theorem}
\begin{proof}

\end{proof}


\begin{coro}\label{corAPM1} Under the Conditions (C!)-(C3) the
equation (\ref{eq2.3}) has a unique almost periodic solution.
\end{coro}

\section{Convergent nonautonomous lattice dynamical systems}

Let $\langle W,\varphi,(Y,\mathbb T,\sigma)\rangle$ (or shortly
$\varphi$) be a cocycle over dynamical system $(Y,\mathbb
T,\sigma)$ with the fiber $W$.

\begin{definition}\label{defCS1} A cocycle $\langle W,\varphi,(Y,\mathbb
T,\sigma)\rangle$ wit the compact base space $Y$ is said to be
convergent if the following conditions are fulfilled:
\begin{enumerate}
\item the cocycle $\varphi$ admits a compact global attractor
$\mathcal I =\{I_{y}|\ y\in Y\}$; \item for any $y\in Y$ the set
$I_{y}$ consists of a single point $\{w_{y}\}$, i.e.,
$I_{y}=\{w_y\}$.
\end{enumerate}
\end{definition}


\begin{theorem}\label{thCGA_2} Under the Conditions
(\textbf{C1})-(\textbf{C3}) the equation (\ref{eq2.3}) (the
cocycle $\varphi$ generated by the equation (\ref{eq2.3})) is
convergent, i.e., it has a compact global attractor $\mathcal I
=\{I_{g}|\ g\in H(f)\}$ such that for any $g\in H(f)$ the set
$I_{g}$ consists of a single point.
\end{theorem}
\begin{proof}

\end{proof}

\section{Funding}

This research was supported by the State Program of the Republic
of Moldova "Monotone Nonautonomous Dynamical Systems
(24.80012.5007.20SE)" and partially was supported by the
Institutional Research Program 011303 "SATGED", Moldova State
University.

\section{Conflict of Interest}

The authors declare that the have not conflict of interest.




\begin{thebibliography}{100}

\bibitem{BLW_2001} Petr W. Bates, Kening Lu and Bixiang Wang,
\newblock Attractors for Lattice Dynamical Systems.
\newblock International Journal of Bifurcation
and Chaos, Vol. 11, No. 1 (2001), pp.143-153.

%\bibitem{Bur} Burbaki N.,
%\newblock{\em Function d'une variable
%r\'{e}elle. Theorie \'{e}l\'{e}mentaire.}
%\newblock Paris,       .

\bibitem{Che_2015} Cheban D. N.
\newblock{\em Global Attractors of Nonautonomous Dynamical and Control
Systems. 2nd Edition.}
\newblock Interdisciplinary Mathematical Sciences,
vol.18, River Edge, NJ: World Scientific, 2015, xxv+589 pp.


\bibitem{Che_2020} David N. Cheban,
\newblock{\em Nonautonomous Dynamics: \emph{Nonlinear Oscillations and
Global Attractors}.}
\newblock{\em Springer Nature Switzerland AG 2020,} xxii+ 434 pp.

\bibitem{Che_2024} David N. Cheban,
\newblock Monotone Nonautonomous Dynamical Systems.
\newblock{\em Springer Nature Switzerland AG,} 2024, xix+460 pp.

\bibitem{CS} Cheban D. N. and Schmalfuss B.,
\newblock Invariant Manifolds, Global Attractors,
Almost Automrphic and Almost Periodic Solutions of Non-Autonomous
Differential Equations.
\newblock{\em Journal of Mathematical Analysis and Applications,} \textbf{340}, no.1 (2008), 374-393.

\bibitem{CS_2025} David Cheban and Andrei Sultan,
\newblock Compact Global Attractors of Nonautonomous Lattice Dynamical Systems.
\newblock{\em  Buletinul Academiei de Stiinte a
Republicii Moldova. Matematica} (to be submitted in 2025).

%\bibitem{DK_1970}
%Daletskii Yu. L. and Krein M. G.,
%\newblock {\em Stability of Solutions of Differential Equations in Banach
%Space}.
%\newblock Moscow, "Nauka", 1970, 534 pp. [English transl., Amer. Math. Soc., Providence, RI
%1974.]

\bibitem{HK_2023} Xiaoying Han and Peter Kloeden,
\newblock Dissipative Lattice Dynamical systems.
\newblock World Scientific, Singapoor, 2023, xv+364 pp.


\bibitem{Lev-Zhi} Levitan B. M. and Zhikov V. V.,
\newblock{\em Almost Periodic Functions and Differential
    Equations.}
\newblock Moscow State University Press, Moscow, 1978 (in Russian).
[English translation: Almost Periodic Functions and Differential
Equations. Cambridge Univ. Press, Cambridge, 1982]


%\bibitem{Sel_1971}
%%Sell G. R.,
%\newblock  Lectures on Topological Dynamics and Differential Equations,
%{\bf vol.2} of {\it Van Nostrand Reinhold math. studies}.
%\newblock Van Nostrand--Reinbold, London, 1971.

%\bibitem{sib}
%K. S. Sibirsky,
%\newblock {\em Introduction to Topological Dynamics.\/}
%\newblock Kishinev, RIA AN MSSR, 1970, 144 p. (in Russian). [English
%translationn: Introduction to Topological Dynamics. Noordhoff,
%Leyden, 1975]

%\bibitem{hartman}
%P. Hartman, On stability in the large for systems of ordinary differential equations.
%Can. J. Math. 13 (1961), 480-492

\end{thebibliography}

\end{document}
