%%%%%% Latex file
%%                              Global Attractors of Non-autonomous Lattice Dynamical Systems
%%                                             David Cheban and Andrei Sultan
%%                                                    November, 2024
%%%%%%%%%%%%%%%%%%%%%%%%%%%%%%


%\usepackage{showkeys}
%\usepackage{amscd}
%\usepackage{latexsym}
%\usepackage{oldlfont}
%\usepackage{latexsym}
%\usepackage{oldlfont}
%\renewcommand{\baselinestretch}{1.5}-
%\input amssym.def
%\input amssym
%% Title
%\title[]{%
% First author
% Second author

\documentclass{amsart}%
\usepackage{amsmath}
\usepackage{amsfonts}
\usepackage{amssymb}
\usepackage{graphicx}%
\usepackage{xcolor}
\usepackage{hyperref}
\setcounter{MaxMatrixCols}{30}
%TCIDATA{OutputFilter=latex2.dll}
%TCIDATA{Version=5.00.0.2552}
%TCIDATA{LastRevised=Wednesday, January 09, 2008 22:34:42}
%TCIDATA{<META NAME="GraphicsSave" CONTENT="32">}
%TCIDATA{<META NAME="SaveForMode" CONTENT="1">}
%TCIDATA{Language=American English}
\def\eps{\varepsilon}
\newtheorem{lemma}{Lemma}[section]
\newtheorem{theorem}[lemma]{Theorem}
\newtheorem{remark}[lemma]{Remark}
\newtheorem{prop}[lemma]{Proposition}
\newtheorem{coro}[lemma]{Corollary}
\newtheorem{definition}[lemma]{Definition}
\newtheorem{example}[lemma]{Example}
\renewcommand{\labelenumi}{(\roman{enumi})}
\parindent0.0em
\parskip0.7em
%\email[T.~Caraballo]{caraball@us.es}
%\email[D.Cheban]{cheban@usm.md}

\title{Global Attractors of Non-autonomous Lattice Dynamical Systems}

% First author
\author{David Cheban}
\address[D. Cheban]{State University of Moldova\\
Vladimir Andrunachievici Instiy=tute of Mathematics and Computer Science\\
Laboratory of Differential Equations\\
A. Mateevich Street 60\\
MD--2009 Chi\c{s}in\u{a}u, Moldova} \email[D.
Cheban]{david.ceban@usm.md, davidcheban@yahoo.com}
% Second author
\author{Andrei Sultan}
\address[A. Sultan]{%
State University of Moldova\\Vladimir Andrunachievici Institute of
Mathematics and Computer Science\\ Laboratory of Differential Equations\\A. Mateevich Street 60\\
MD--2009 Chi\c{s}in\u{a}u, Moldova} \email[A.
Sultan]{andrew15sultan@gmail.com}





\date{\today}
\subjclass{34D05, 34D45, 34G20, 37B55} \keywords{Lattice Dynamical
Systems; Non-autonomous Dynamical Systems; Cocycles}

%%%%%%%%%%%%%%%%%%%%%%%%%%%%%%
%%%%%%%%%%%%%%%%%%%%%%%%%%%%%%
%%%%%%%%%%


\begin{document}

\begin{abstract}
The aim of this paper is studying the compact global attractors
for non-autonomous lattice dynamical systems of the form
$u_{i}'=\nu (u_{i-1}-2u_i+u_{i+1})-\lambda u_{i}+f(u_i)+f_{i}(t)\
(i\in \mathbb Z,\ \lambda >0)$. We prove their dissipativness,
asymptotic compactness and then the existence of compact global
attractors.
\end{abstract}

\maketitle



\section{Introduction}\label{Sec1}

Denote by $\mathbb R :=(-\infty,\infty)$, $\mathbb Z :=\{0,\pm
1,\pm 2,\ldots\}$ and $\ell_{2}$ the Hilbert space of all
two-sided sequences $\xi =(\xi_{i})_{i\in \mathbb Z}$ ($\xi_{i}\in
\mathbb R$) with
\begin{equation}\label{eqI_1}
\sum\limits_{i\in \mathbb Z}|\xi_{i}|^{2}<+\infty
\end{equation}
and equipped with the scalar product
\begin{equation}\label{eqI_2}
\langle \xi,\eta\rangle :=\sum\limits_{i\in \mathbb
Z}\xi_{i}\eta_{i} .
\end{equation}
Let $(\mathfrak B, |\cdot|)$ be a Banach space with the norm
$|\cdot|$, $C(\mathbb R,\mathfrak B)$ be the space of all
continuous functions $f:\mathbb R\to \mathfrak B$ equipped with
the distance
\begin{equation}\label{eqI_3}
d(f_1,f_2):=\sup\limits_{L>0}\min\{\max\limits_{|t|\le
L}|f_{1}(t)-f_{2}(t)|,L^{-1}\}.
\end{equation}
The metric space $(C(\mathbb R,\mathfrak B),d)$ is complete and
the distance $d$, defined by (\ref{eqI_3}), generates on the space
$C(\mathbb R,\mathfrak B)$ the compact-open topology.

Let $h\in \mathbb R$, $f\in C(\mathbb R,\mathfrak B)$,
$f^{h}(t):=f(t+h)$ for any $t\in \mathbb R$ and $\sigma :\mathbb
R\times C(\mathbb R,\mathfrak B)\to C(\mathbb R,\mathfrak B)$ be a
mapping defined by $\sigma(h,f):=f^{h}$ for any $(h,f)\in \mathbb
R\times C(\mathbb R,\mathfrak B)$. Then \cite[Ch.I]{Che_2015} the
triplet $(C(\mathbb R,\mathfrak B),\mathbb R,\sigma)$ is a shift
dynamical system (or Bebutov's dynamical system) on he space
$C(\mathbb R,\mathfrak B)$. By $H(f)$ the closure in the space
$C(\mathbb R,\mathfrak B)$ of $\{f^{h}|\ h\in \mathbb R\}$ is
denoted.
{\color{red}
compare results pullback attractors
}
In this note we study the compact global attractors of the systems
\begin{equation}\label{eqI1}
u_{i}'=\nu (u_{i-1}-2u_i+u_{i+1})-\lambda u_{i}+F(u_i)+f_{i}(t)\
(i\in \mathbb Z),
\end{equation}
where $\lambda >0$, $F\in C(\mathbb R, \mathbb R)$ and $f\in
C(\mathbb R,\ell_{2})$ ($f(t):=(f_{i}(t))_{i\in \mathbb Z}$ for
any $t\in \mathbb R$).

The system (\ref{eqI1}) can be considered as a discrete (see, for
example, \cite{BLW_2001}, \cite{HK_2023} and the bibliography
therein) analogue of a reaction-diffusion equation in $\mathbb R$:
\begin{equation}\label{eqI1.1}
 \frac{\partial{u}}{\partial{t}} = D\frac{\partial^{2}{u}}{\partial^{2}{x}}-\lambda u + F(u) + f(t,x),
\end{equation}
where grid points are spaced $h$ distance apart and $\nu =
D/h^{2}$.

This study continues the first author's works devoted to the study
of compact global attractors of non-autonomous dynamical systems
\cite{Che_2015} and compact attractors of lattice dynamical
systems \cite{BLW_2001} (autonomous systems) and compact pullback
attractors \cite{HK_2023} (for non-autonomous systems).

The paper is organized as follows. In the second section we show
that under some conditions the equation (\ref{eqI1}) generates a
cocycle which plays a very important role in the study of the
asymptotic properties of the equation (\ref{eqI1}). In the third
section we prove that under some conditions the existence of an
absorbing set for the equation (\ref{eqI1}). The fourth section is
dedicated to the study the asymptotically compactness of the
cocycle generated by the equation (\ref{eqI1}). In the fifth
section we study the problem of existence of a compact global
attractor for the equation (\ref{eqI1}).


\section{Cocycles}\label{Sec2}

Consider a non-autonomous system
\begin{equation}\label{eq2.1}
u_{i}'=\nu (u_{i-1}-2u_i+u_{i+1})-\lambda u_{i}+F(u_i)+f_{i}(t)\
(i\in \mathbb Z) .
\end{equation}

Below we use the following conditions.

\emph{Condition }(\textbf{C1}). \label{C1}
The function $f\in C(\mathbb
R,\mathfrak B)$ is translation-compact, i.e., the set $\{f^{h}|\
h\in \mathbb R\}$ is pre-compact in the space $C(\mathbb
R,\mathfrak B)$.

\begin{lemma}\label{l2.1} \cite{Sel_1971,sib} The following statements are equivalent:
\begin{enumerate}
\item the function $f\in C(\mathbb R,\mathfrak B)$ is
translation-compact; \item the set $Q:=\overline{f(\mathbb R)}$ is
compact in $\mathfrak B$ and the function $f\in C(\mathbb
R,\mathfrak B)$ is uniformly continuous.
\end{enumerate}
\end{lemma}

\emph{Condition} (\textbf{C2}). \label{C2}
The function $F\in C(\mathbb
R,\mathbb R)$ is continuously differentiable, \(F'(x)\)
is globally bounded in \(\mathfrak L\)
 and $F(0)=0$. Denote by $
\widetilde{F}:\ell_{2}\to \ell_{2}$ the Nemytskii operator
generated by $F$, i.e., $\widetilde{F}(\xi)_{i}:=F(\xi_{i})$ for
any $i\in \mathfrak N$.

{\color{violet} 
\emph{Condition} (\textbf{C2.1}). \label{C2.1}
The function $F\in C(\mathbb
R,\mathbb R)$  is Lipschitz continuous on bounded sets.

\emph{Condition} (\textbf{C2.2}). \label{C2.2}
\(sF(s)\leq -\alpha s^2\).
}

\begin{definition}\label{defL1.8} A function $F\in C(Y\times \mathfrak B,\mathfrak
B)$ is said to be globally Lipschitzian (respectively locally Lipschitzian) with respect to variable
$u\in \mathfrak B$ uniformly with respect to $y\in Y$ if there
exists a positive constant $L$ (for any bounded set \( \mathfrak B \subset \mathfrak L\) there exists a constant \(L_{\mathfrak B}\)) such that
\begin{equation}\label{eqL1.81}
|F(y,u_1)-F(y,u_2)|\le L |u_1-u_2|
\end{equation}
Respectively, 
\begin{equation}\label{eqL1.82}
|F(y,v_1)-F(y,v_2)|\le L_{\mathfrak B} |v_1-v_2|
\end{equation}
for any $u_1,u_2\in \mathfrak B$ and $y\in Y$ (respectively \(v_1, v_2 \in \mathfrak B \subset \mathfrak L\)).
\end{definition}

\begin{definition}\label{defL2.8}
The smallest constant $L$ (respectively \(L_{\mathfrak B}\)) with the property (\ref{eqL1.81}) is
called Lipshchitz constant of function $F$ (notation $Lip(F)$, respectively \(Lip_{\mathfrak B}(F)\)).
\end{definition}

Let \(\mathfrak G \subset \mathfrak L\), denote by $CL(Y\times \mathfrak G,\mathfrak B)$
 the Banach space of any 
Lipschitzian functions $F\in C(Y\times \mathfrak G,\mathfrak B)$
equipped with the norm
\begin{equation}
||F||_{CL}:=\max\limits_{y\in Y}|F(y,0)|+Lip_{\mathfrak G}(F).\nonumber
\end{equation}

\begin{lemma}\label{l2.2}\cite{Che_2015} Under the Condition (\textbf{C2})
it is well defined the mapping $\widetilde{F}:\ell_{2}\to
\ell_{2}$ and
\begin{equation}\label{eq2.2}
|\widetilde{F}(\xi)-\widetilde{F}(\eta)|\le Lip_{\mathfrak B}(F)|\xi -\eta|
\end{equation}
for any $\xi,\eta\in \ell_{2}$.
\end{lemma}

For any $u = (u_{i})_{i\in \mathbb Z}$, the discrete Laplace
operator $\Lambda$ is defined \cite[Ch.III]{HK_2023} from
$\ell_{2}$ to $\ell_{2}$ component wise by $\Lambda(u)_{i} =
u_{i-1} - 2u_{i} + u_{i+1}$ ($i\in \mathbb Z$). Define the
bounded linear operators $D^{+}$ and $D^{-}$ from $\ell_{2}$ to
$\ell_{2}$ by $(D^{+}u)_{i} = u_{i+1} - u_{i},\ (D^{-}u)_{i} =
u_{i-1} - u_{i}\ (i\in \mathbb Z)$.

Note that $\Lambda = D^{+}D^{-} = D^{-}D^{+}$ and $\langle D^{-}u,
v\rangle = \langle u, D^{+}v\rangle $ for any $u,v\in \ell_{2}$
and, consequently, $\langle \Lambda u,u \rangle = -|D^{+}u|^{2}\le
0$. Since $\Lambda$ is a bounded linear operator acting on the
space $\ell_{2}$, it generates a uniformly continuous semi-group
on $\ell_{2}$.

Under the Conditions (C1) and (C2) the system of differential
equations (\ref{eq2.1}) can be written in the form of an ordinary
differential equation
\begin{equation}\label{eq2.3}
u'=\nu \Lambda u +\Phi (u)+f(t)
\end{equation}
in the Banach space $\mathfrak B=\ell_{2}$, where
$\Phi(u):=-\lambda u +\widetilde{F} (u)$ and
$\Lambda(u)_{i}:=u_{i-1}-2u_{i}+u_{i+1}$ for any $u=(u_i)_{i\in
\mathbb Z}\in \ell_{2}$. Along with equation (\ref{eq2.3}) we
consider also it $H$-class, i.e., the family equations
\begin{equation}\label{eq2.3g}
u'=\nu \Lambda u +\Phi (u)+g(t),
\end{equation}
where $g\in H(f)$.

The family of equations (\ref{eq2.3g}) can be rewritten as follows
\begin{equation}\label{eq2.3H}
u'=F(\sigma(t,g),u)\ \ (g\in H(f)),
\end{equation}
where $F:H(f)\times \ell_{2}\to \ell_{2}$ is defined by
$F(g,u):=\nu \Lambda u+\Phi (u) +g(0)$. It easy to see that
$F(\sigma(t,g),u)=\nu \Lambda u+\Phi (u)+g(t)$ for any $(t,u,g)\in
\mathbb R\times \mathfrak B \times H(f)$.

Let $(Y,\mathbb R,\sigma)$ be a dynamical system on the metric
space $Y$.

\begin{lemma}\label{l2.3}\cite{Che_2015} Assume that the function
$\mathcal F \in C(\mathbb R\times \mathfrak B,\mathfrak B)$
satisfies the (global) Lipschitz condition with the Lipschitz
constant $L_{\mathcal F}$, then every function $\mathcal G\in
H(\mathcal F)$ satisfies the (global) Lipschitz condition with the
Lipschitz constant $L_{\mathcal G}\le L_{\mathcal F}$.
\end{lemma}

 Let $Y$ be a complete metric space,
$(Y,\mathbb R,\sigma)$ be a dynamical system on $Y$ and $ \Lambda
$ be some complete metric space of linear closed operators acting
into Banach space $ \mathfrak B $. Consider the following linear
differential equation
\begin{equation}\label{eqLS01.81}
x'=A(\sigma(t,y))x,\  \ (y\in Y)
\end{equation}
where $A\in C(Y,\Lambda)$. We assume that the following conditions
are fulfilled for equation (\ref{eqLS01.81}):
\begin{enumerate}
\item[a.] for any $ u \in \mathfrak B $ and $y\in Y $ equation
(\ref{eqLS01.81}) has exactly one solution that is defined on $
\mathbb R_{+} $ and satisfies the condition $ \varphi (0,u,y) = u
;$ \item[b.] the mapping $ \varphi : (t,u,y) \to \varphi (t,u,y) $
is continuous in the topology of $ \mathbb R_{+} \times \mathfrak
B \times Y$.
\end{enumerate}

Denote by $U(t,y):=\varphi(t,\cdot,y)$ for any $(t,y)\in \mathbb
R_{+}\times Y$.


Consider an evolutionary differential equation
\begin{equation}\label{eqSL1}
u'=A(\sigma(t,y))u + F(\sigma(t,y),u) \ \ (y\in Y)
\end{equation}
in the Banach space $\mathfrak B$, where $F$ is a nonlinear
continuous mapping ("small" perturbation) acting from $Y\times
\mathfrak B$ into $\mathfrak B$.

\begin{definition}
A function $u:[0,a)\mapsto \mathfrak B$ is said to be a weak
(mild) solution of equation (\ref{eqSL1}) passing through the
point $x\in \mathfrak B$ at the initial moment $t=0$ (notation
$\varphi(t,x,y)$) if $u\in C([0,T],\mathfrak B)$ and satisfies the
integral equation
\begin{equation}\label{eqSL3}
u(t)=U(t,y)x+\int_{0}^{t}U(t-s,\sigma(s,y))F(\sigma(s,y),u(s))ds
\end{equation}
for any $t\in [0,T]$ and $0<T<a$.
\end{definition}


\begin{theorem}\label{thLS2.8} \cite[Ch.VI]{Che_2020}
Suppose that the function $F\in C(Y\times \mathfrak B,\mathfrak
B)$ is globally Lipschitzian\index{globally Lipschitzian
function}.

Then for any $(x,y)\in \mathfrak B\times Y$ there exists a unique
solution $\varphi(t,x,y)$ of the equation (\ref{eqSL1}) defined on
the semi-axis $[0,+\infty)$ with the conditions: $\varphi
(0,x,y)=x$ and the mapping $ \varphi : [0,+\infty)\times \mathfrak
B \times Y \to \mathfrak B\ ( (t,x,y)\mapsto \varphi (t,x,y))$ is
continuous.
\end{theorem}

\begin{theorem}\label{th1.1} Under the Conditions (\textbf{C1}) and
(\textbf{C2}) the following statements hold:
\begin{enumerate}
\item for any $(v,g)\in \ell_{2}\times H(f)$ there exists a unique
solution $\varphi(t,v,g)$ of the equation (\ref{eq2.3g}) passing
through the point $v$ at the initial moment $t=0$ and defined on
the semi-axis $\mathbb R_{+}:=[0,+\infty)$; \item
$\varphi(0,v,g)=v$ for any $(v,g)\in \ell_{2}\times H(f)$; \item
$\varphi(t+\tau,v,g)=\varphi(t,\varphi(\tau,v,g),g^{\tau})$ for
any $t,\tau\in \mathbb R_{+}$, $v\in \ell_{2}$ and $g\in H(f)$;
 \item the mapping
$\varphi :\mathbb R_{+}\times \ell_{2}\times H(f)\to \ell_{2}$
($(t,v,g)\to \varphi(t,v,g))$ for any $(t,v,g)\in \mathbb
R_{+}\times \ell_{2}\times H(f)$) is continuous.
\end{enumerate}
\end{theorem}
\begin{proof}
Assume that the Conditions (\textbf{C1}) and (\textbf{C2}) are
fulfilled. Consider the equation (\ref{eq2.3H}), where
$F(g,u):=\nu \Lambda u +\Phi(u)+g(0)$ for any $(u,g)\in
\ell_{2}\times H(f)$. It easy to check that under the conditions
of Theorem the mapping $F$ possesses the following properties:
\begin{enumerate}
\item $F$ is continuous; \item the mapping $F$ is Lipschitzian in
$u\in \ell_{2}$ uniformly with respect to $g\in H(f)$ with the
Lipschitz constant $L_{F}\le L_{\mathcal F}$; \item there exists a
positive constant $A$ such that
\begin{equation}\label{eqA1}
|F(g,0)|\le A
\end{equation}
for any $g\in H(f)$.
\end{enumerate}

Now to finish the proof of Theorem it suffices to apply Theorem
\ref{thLS2.8}.
\end{proof}

{\color{violet}
\begin{lemma} \textit{Assume that (\hyperref[C1]{C1}) and (\hyperref[C2.2]{C2.2}) holds and $g \in H(f)$. Then, for every $T > 0$, any solution $u$ of the problem (\ref{eqI1}) and (\(u(0)=u_0 \in \ell^2\)) satisfies}
\[
\|u(t)\| \le C \;,\quad \text{for all } 0 \le t \le T\,,
\]
\textit{where $C$ is a constant depending only on the data $(\lambda, \|f\|, \|u_0\|)$ and $T$.}

\textit{Proof.} \; Taking the inner product of (\ref{eqI1}) with $u$ in $\ell^2$, by (6) and (\ref{eqL1.82}) we find that
\[
\frac{1}{2} \frac{d}{dt} \|u\|^2 + \nu \|Bu\|^2 + \lambda \|u\|^2 = -(f(u), u) + (g(t), u)\,.
\tag{l1.1}
\]
Since
\[
|(g(t),u)| \le \|g(t)\|\|u\| \le \frac{1}{2}\lambda \|u\|^2 + \frac{1}{2\lambda}\|g(t)\|^2\,,
\]
using (C2.2) we get that
\[
\frac{d}{dt}\|u\|^2 + 2\nu\|Bu\|^2 + \lambda\|u\|^2 \le \frac{1}{\lambda}\|g(t)\|^2\, \leq \frac{C}{\lambda}
\tag{l1.2}
\]
Then Lemma 2.1 follows from (l1.2) and Gronwall's
lemma. 
\end{lemma}
}

{\color{violet}
\begin{theorem}\cite[Ch.VI]{Che_2020} \cite[Ch.II]{hartman}
\label{th1.2} Under the Conditions (\hyperref[C1]{C1}) and
(\hyperref[C2.1]{C2.1}) the following statements hold:
\begin{enumerate}
    \item for any $(v,g)\in \ell_{2}\times H(f)$ there exists a unique
    solution $\varphi(t,v,g)$ of the equation (\ref{eq2.3g}) passing
    through the point $v$ at the initial moment $t=0$ and defined on
    the semi-axis $\mathbb R_{+}:=[0,+\infty)$; \item
    $\varphi(0,v,g)=v$ for any $(v,g)\in \ell_{2}\times H(f)$; \item
    $\varphi(t+\tau,v,g)=\varphi(t,\varphi(\tau,v,g),g^{\tau})$ for
    any $t,\tau\in \mathbb R_{+}$, $v\in \ell_{2}$ and $g\in H(f)$;
     \item the mapping
    $\varphi :\mathbb R_{+}\times \ell_{2}\times H(f)\to \ell_{2}$
    ($(t,v,g)\to \varphi(t,v,g))$ for any $(t,v,g)\in \mathbb
    R_{+}\times \ell_{2}\times H(f)$ is continuous.
\end{enumerate}
\end{theorem}
\begin{proof}
Assume that the Conditions (\textbf{C1}) and (\textbf{C2.1}) are
fulfilled. Consider the equation (\ref{eq2.3H}), where
$F(g,u):=\nu \Lambda u +\Phi(u)+g(0)$ for any $(u,g)\in
\ell_{2}\times H(f)$. It easy to check that under the conditions
of Theorem the mapping $F$ possesses the following properties:
\begin{enumerate}
\item $F$ is continuous; 
\item the mapping $F$ is locally Lipschitzian in
$u\in \ell_{2}$ uniformly with respect to $g\in H(f)$ with the
Lipschitz constant $L_{F}\le L_{\mathcal F}$; 
\item there exists a
positive constant $A$ such that
\begin{equation}\label{eqA2}
|F(g,0)|\le A
\end{equation}
for any $g\in H(f)$.
\end{enumerate}

Now to finish the proof of Theorem it suffices to apply Lemma 2.9 and \cite[Ch.II]{hartman}
\ref{thLS2.8}.
\end{proof}
}

Let $Y$ be a complete metric space and $(Y,\mathbb R,\sigma)$ be a
dynamical system on $Y$.

\begin{definition}\label{defC1} Recall \cite[Ch.I]{Che_2015} that
$\langle \mathbb B,\varphi, (Y,\mathbb R,\sigma)\rangle$ is said
to be a cocycle over $(Y,\mathbb R,\sigma)$ with the fiber
$\mathfrak B$ if $\varphi$ is a continuous mapping acting from
$\mathbb R_{+}\times \mathfrak B\times Y\to \mathfrak B$
satisfying the following conditions:
\begin{enumerate}
\item $\varphi(0,u,y)=v$ for any $(v,y)\in \mathfrak B\times Y$;
\item
$\varphi(t+\tau,u,y)=\varphi(t,\varphi(\tau,u,t),\sigma(\tau,y))$
for any $(t,\tau \in \mathbb R_{+}$ and $(u,)\in \mathfrak B\times
Y$.
\end{enumerate}
\end{definition}

\begin{coro}\label{corH1}
Under the conditions of Theorem \ref{th1.1} the equation
(\ref{eq2.3}) (respectively, the family of equations
(\ref{eq2.3g})) generates a cocycle $\langle
\ell_{2},\varphi,(H(f),\mathbb R,\sigma)\rangle$ over the shift
dynamical system $(H(f),\mathbb R,\sigma)$ with the fiber
$\ell_{2}$.
\end{coro}
\begin{proof} This statement directly follows from Theorem
\ref{th1.1} and Definition \ref{defC1}.
\end{proof}


\section{Existence of an absorbing set}\label{Sec3}

\emph{Condition} (\textbf{C3}). There exist positive numbers
$\alpha$ and $\gamma$ such that $F(s)s\le -\alpha s^{2}+\gamma$
for any $s\in \mathbb R$.

\begin{theorem}\label{th2.1} Under the Conditions (\textbf{C1})-(\textbf{C3})
there exists a closed ball $B[0,r]:=\{\xi \in \ell_{2}|\ |\xi|\le
r\}$ such that for any bounded subset $B\subset \ell_{2}$ there
exist a positive number $L=L(B)$ such that
$\varphi(t,B,Y)\subseteq B[0,r]$ for any $t\ge L(B)$, where
$\varphi(t,M,Y):=\{\varphi(t,u,y)|\ u\in M,\ y\in Y\}$.
\end{theorem}

{\color{violet}
\begin{lemma}[Gronwall]
    If \(y'\leq -\alpha y + \beta, \alpha>0,\beta>0 \Rightarrow  y(t) = y(0)e^{-\alpha t}+\frac{\beta}{\alpha}(1-e^{-\alpha t}) \)
\end{lemma}
Proof of \ref{th2.1}:
Taking the inner product of equation (10) with \(u\) gives:

% \[\frac{d}{dt}\left\lVert u\right\rVert ^2 = 2\nu \left\langle \Lambda u, u\right\rangle + 2 \left\langle \Phi (u), u\right\rangle + 2 \left\langle g(t), u\right\rangle \leq 
% 2 \sum_{i \in \mathbb{Z} } u_i f(u_i) + 2 \sum_{i \in \mathbb{Z} } g(t)_i u_i \leq
% -\alpha \left\lVert u\right\rVert ^2 + \frac{\left\lVert g\right\rVert ^2}{\alpha} \leq
% -\alpha \left\lVert u\right\rVert ^2 + \frac{C(g)}{\alpha} \leq
% \]
\begin{align}
    \frac{d}{dt}\left\lVert u\right\rVert ^2 &= 2\nu \left\langle \Lambda u, u\right\rangle + 2 \left\langle \Phi (u), u\right\rangle + 2 \left\langle g(t), u\right\rangle \notag \\
    &\leq 2 \sum_{i \in \mathbb{Z} } u_i f(u_i) + 2 \sum_{i \in \mathbb{Z} } g(t)_i u_i \notag \\
    &\leq -\alpha \left\lVert u\right\rVert ^2 + \frac{\left\lVert g\right\rVert ^2}{\alpha} \notag \\
    &\leq -\alpha \left\lVert u\right\rVert ^2 + \frac{C(g)}{\alpha} \leq -\alpha \left\lVert u\right\rVert ^2 + \frac{C}{\alpha}
\end{align}
where the penultimate step follows from Young’s inequality, and \(C(g) \leq C\) since \(\all g \in H(f)\). Hence, Gronwall’s lemma
implies that
\[
\left\lVert u(t) \right\rVert^2 \leq \left\lVert u_0 \right\rVert^2 e^{-\alpha t} + \frac{\left\lVert C \right\rVert^2}{\alpha^2} \left( 1 - e^{-\alpha t} \right), \quad t \ge 0.
\]
Define the closed ball \( Q \) in \( \ell^2 \) by
\[
Q := \left\{ u \in \ell^2 : \left\lVert u \right\rVert^2 \le R^2 := 1 + \frac{\left\lVert C \right\rVert^2}{\alpha^2} \right\}.
\]

}



\section{Asymptotically compactness of the cocycle generated by
the equation (\ref{eqI1})}\label{Sec4}

Let $\langle \mathfrak B,\varphi, (Y,\mathbb R,\sigma)\rangle$ (or
shortly $\varphi$) be a cocycle over dynamical system $(Y,\mathbb
R,\sigma)$ with the compact phase space $Y$.

Let $A$ and $B$ be two bounded subsets from $\mathfrak B$. Denote
by $\rho(a,b):=|a-b|$ ($a,b\in \mathfrak B$),
$\rho(a,B):=\inf\limits_{b\in B}\rho(a,b)$ and
\begin{equation}\label{eqAC1}
\beta(A,B):=\sup\limits_{a\in A}\rho(a,B).
\end{equation}

\begin{definition}\label{defAC1} A cocycle $\varphi$ is said to be
asymptotically compact if for any bounded subset $B\subset \
\mathfrak B$ there exists a compact subset $K=K(B)\subset
\mathfrak B$ such that the compact subset $K$ attracts the bounded
set $B$, that is,
\begin{equation}\label{eqAC2}
\lim\limits_{t\to +\infty}\sup\limits_{y\in
Y}\beta(\varphi(t,B,y),K)=0 .
\end{equation}
\end{definition}

\begin{lemma}\label{lAC1} \cite[Ch.I]{Che_2015} Assume that $Y$ is a compact metric space and
$\varphi$ is a cocycle over dynamical system $(Y,\mathbb
R,\sigma)$ with the fiber $\mathfrak B$.

Then the following statements are equivalent:
\begin{enumerate}
\item the cocycle $\varphi$ is asymptotically compact; \item from
any sequence $\{\varphi(t_n,u_n,y_n)\}_{n\in \mathbb N}$ (with
bounded $\{u_n\}\subset \mathfrak B$ and $t_n\to +\infty$ as $n\to
\infty$) can be extracted a convergent subsequence
$\{\varphi(t_{k_n},u_{k_n},y_{k_n})\}$.
\end{enumerate}
\end{lemma}

\begin{theorem}\label{thAC1} Under the Conditions
(\textbf{C1})-(\textbf{C3}) the cocycle $\langle
\ell_{2},\varphi,(H(f),\mathbb R,\sigma)\rangle$ generated by the
equation (\ref{eq2.3}) is asymptotically compact.
\end{theorem}
{\color{violet}
We will prove this statement using the ideas and methods elaborated in the work (\cite{HK_2023}, Chapter 3.2).
Consider a smooth function \( \xi : \mathbb{R}^+ \to [0, 1] \) satisfying
\[
\xi(s) =
\begin{cases}
0, & 0 \le s \le 1, \\
\in [0, 1], & 1 \le s \le 2, \\
1, & s \ge 2
\end{cases}
\]

and note that there exists a constant \( C_0 \) such that \( |\xi'(s)| \le C_0 \) for all \( s \ge 0 \).
Then for a fixed \( k \in \mathbb{N} \) (its value will be specified later), define
\[
\xi_k(s) = \xi \left( \frac{s}{k} \right) \quad \text{for all} \quad s \in \mathbb{R_+}.
\]

Given \( \mathbf{u} \in \ell^2 \), define \( \mathbf{v} \in \ell^2 \) componentwise as
\[
v_i := \xi_k( |i| ) u_i \quad \text{for} \quad i \in \mathbb{Z}.
\]

Taking the inner product of equation (\ref{eq2.3g}) with \( \mathbf{v} \) gives
\[
\frac{d}{dt} \langle \mathbf{u}, \mathbf{v} \rangle + \nu \langle \mathrm{D}^+ \mathbf{u}, \mathrm{D}^+ \mathbf{v} \rangle = \langle \Phi(\mathbf{u}), \mathbf{v} \rangle + \langle \mathbf{g(t)}, \mathbf{v} \rangle,
\]

that is
\begin{equation}\label{20}
\frac{d}{dt} \sum_{i \in \mathbb{Z}} \xi_k( |i| ) |u_i|^2 + 2 \nu \langle \mathrm{D}^+ \mathbf{u}, \mathrm{D}^+ \mathbf{v} \rangle = 2 \sum_{i \in \mathbb{Z}} \xi_k( |i| ) u_i f(u_i) + 2 \sum_{i \in \mathbb{Z}} \xi_k( |i| ) g(t)_i u_i
\end{equation}

Each term in equation (\ref{20}) will now be estimated. First,
\[
\left\langle \mathrm{D}^+ \mathbf{u}, \mathrm{D}^+ \mathbf{v} \right\rangle = \sum_{i \in \mathbb{Z}} (u_{i+1} - u_i)(v_{i+1} - v_i)
\]
\[
= \sum_{i \in \mathbb{Z}} (u_{i+1} - u_i) \left[ \left( \xi_k(|i+1|) - \xi_k(|i|) \right) u_{i+1} + \xi_k(|i|)(u_{i+1} - u_i) \right]
\]
\[
= \sum_{i \in \mathbb{Z}} \left( \xi_k(|i+1|) - \xi_k(|i|) \right) (u_{i+1} - u_i) u_{i+1} + \sum_{i \in \mathbb{Z}} \xi_k(|i|) (u_{i+1} - u_i)^2
\]
\[
\ge \sum_{i \in \mathbb{Z}} \left( \xi_k(|i+1|) - \xi_k(|i|) \right) (u_{i+1} - u_i) u_{i+1}.
\]
Since
\[
\left| \sum_{i \in \mathbb{Z}} \left( \xi_k(|i+1|) - \xi_k(|i|) \right) (u_{i+1} - u_i) u_{i+1} \right| 
\le \sum_{i \in \mathbb{Z}} \frac{1}{k} |\xi'(s_i)| \cdot |u_{i+1} - u_i| \cdot |u_{i+1}|,
\]
for some \( s_i \) between \( |i| \) and \( |i+1| \), and
\[
\sum_{i \in \mathbb{Z}} |\xi'(s_i)| \, |u_{i+1} - u_i| \, |u_{i+1}| 
\le C_0 \sum_{i \in \mathbb{Z}} \left( |u_{i+1}|^2 + |u_i||u_{i+1}| \right) \le 4 C_0 \left\| \mathbf{u} \right\|^2.
\]

Then it follows that for all \( \mathbf{u} \in Q \) and \( \mathbf{v} \in \ell^2 \) defined componentwise as \( v_i := \xi_k(|i|) u_i \), for \( i \in \mathbb{Z} \),
\begin{equation}\label{In20}
\left\langle \mathrm{D}^+ \mathbf{u}, \mathrm{D}^+ \mathbf{v} \right\rangle 
\ge - \frac{4 C_0 \|Q\|^2}{k}.
\end{equation}
where \( \|Q\| := \sup_{\mathbf{u} \in Q} \left\| \mathbf{u} \right\| \).
On the other hand, by Assumption C2.2,
\[
2 \sum_{i \in \mathbb{Z}} \xi_k(|i|) u_i f(u_i) \le -2 \alpha \sum_{i \in \mathbb{Z}} \xi_k(|i|) |u_i|^2
\]

and by Young's inequality
\[
2 \sum_{i \in \mathbb{Z}} \xi_k(|i|) g_i u_i \le \alpha \sum_{i \in \mathbb{Z}} \xi_k(|i|) |u_i|^2 + \frac{1}{\alpha} \sum_{i \in \mathbb{Z}} \xi_k(|i|) |g_i|^2.
\]

Thus
\begin{equation}\label{22}
2 \sum_{i \in \mathbb{Z}} \xi_k(|i|) u_i f(u_i) + 2 \sum_{i \in \mathbb{Z}} \xi_k(|i|) g_i u_i
\le -\alpha \sum_{i \in \mathbb{Z}} \xi_k(|i|) |u_i|^2 + \frac{1}{\alpha} \sum_{|i| \ge k} |g_i|^2
\end{equation}

Using the estimates (\ref{In20}) and (\ref{22}) in the equation (\ref{20}) gives
\begin{equation}\label{EqEst}
\frac{d}{dt} \sum_{i \in \mathbb{Z}} \xi_k(|i|) |u_i|^2 + \alpha \sum_{i \in \mathbb{Z}} \xi_k(|i|) |u_i|^2
\le \nu \frac{4 C_0 \| Q \|^2}{k} + \frac{1}{\alpha} \sum_{|i| \ge k} |g_i|^2
\end{equation}
Because \( \mathbf{g(t)}\) is uniformly limited, for every \( \varepsilon > 0 \), there exists \( K(\varepsilon) \) such that
\[
\nu \frac{4C_0 \|Q\|^2}{k} + \frac{1}{\alpha} \sum_{|i| \ge k} |g(t)_i|^2 \le \varepsilon, \quad k \ge K(\varepsilon), \forall t \in \mathbb{R}.
\]

The inequality (\ref{EqEst}) along with the relation above give
\[
\frac{d}{dt} \sum_{i \in \mathbb{Z}} \xi_k(|i|) |u_i|^2 + \alpha \sum_{i \in \mathbb{Z}} \xi_k(|i|) |u_i|^2 \le \varepsilon.
\]

Then, Gronwall's lemma implies that
\[
\sum_{i \in \mathbb{Z}} \xi_k(|i|) |u_i(t, \mathbf{u}_o)|^2 \le e^{-\alpha t} \sum_{i \in \mathbb{Z}} \xi_k(|i|) |u_{o, i}|^2 + \frac{\varepsilon}{\alpha}
\le e^{-\alpha t} \| \mathbf{u}_o \|^2 + \frac{\varepsilon}{\alpha}.
\]

Hence for every \( \mathbf{u}_o \in Q \),
\[
\sum_{i \in \mathbb{Z}} \xi_k(|i|) |u_i(t, \mathbf{u}_o)|^2 \le e^{-\alpha t} \| Q \|^2 + \frac{\varepsilon}{\alpha}.
\]
and therefore
\[
\sum_{i \in \mathbb{Z}} \xi_k(|i|) \left| u_i(t, \mathbf{u}_o) \right|^2 \le \frac{2\varepsilon}{\alpha}, \quad \text{for } t \ge T(\varepsilon) := \frac{1}{\alpha} \ln \frac{\alpha \| Q \|^2}{\varepsilon}.
\]
}
In conclusion we can state that Th. 4.3 is proved.

\section{Compact global attractors}\label{Sec5}

\begin{definition}\label{defCGA0_1} A family $\{I_{y}|\ y\in Y\}$ of
compact subsets $I_{y}$ of $\mathfrak B$ is said to be a compact
global attractor for the cocycle $\langle \mathfrak
B,\varphi,(Y,\mathbb R,\sigma)\rangle$ if the following conditions
are fulfilled:
\begin{enumerate}
\item the set
\begin{equation}\label{eqCGA1}
\mathcal I :=\bigcup \{I_{y}|\ y\in Y\}
\end{equation}
is precompact; \item the family of subsets $\{I_{y}|\ y\in Y\}$ is
invariant, i.e., $\varphi(t,I_{y},y)=I_{\sigma(t,y)}$ for any
$(t,y)\in \mathbb R_{+}\times Y$; \item
\begin{equation}\label{eqCGA2}
\lim\limits_{t\to +\infty}\sup\limits_{y\in
Y}\beta(\varphi(t,M,y),\mathcal I)=0
\end{equation}
\end{enumerate}
for any compact subset $M$ from $\mathfrak B$.
\end{definition}


\begin{definition}\label{defCGA1} A cocycle $\varphi$ is said to be
dissipative if there exists a bounded subset $K\subset \mathfrak
B$ such that for any bounded subset $B\subset \ \mathfrak B$ there
exists a positive number $L=L(B)$ such that
$\varphi(t,B,Y)\subseteq K$ for any $t\ge L(B)$, where
$\varphi(t,B,Y):=\{\varphi(t,u,y)|\ (u,y)\in B\times Y\}$.
\end{definition}


\begin{theorem}\label{thCGA1} \cite[Ch.II]{Che_2024} Assume that the metric space $Y$ is
compact and the cocycle $\langle \mathfrak B,\varphi,(Y,\mathbb
R,\sigma)\rangle$ is dissipative and asymptotically compact.

Then the cocycle $\varphi$ has a compact global attractor.
\end{theorem}

\begin{theorem}\label{thCGA2} Under the Conditions
(\textbf{C1})-(\textbf{C3}) the equation (\ref{eq2.3}) (the
cocycle $\varphi$ generated by the equation (\ref{eq2.3})) has a
compact global attractor $\{I_{g}|\ g\in H(f)\}$.
\end{theorem}
\begin{proof} This statement follows from Theorems \ref{th2.1},
\ref{thAC1} and \ref{thCGA1}.
\end{proof}

\section{Funding}

This research was supported by the State Program of the Republic
of Moldova "Monotone Nonautonomous Dynamical Systems
(24.80012.5007.20SE)" and partially was supported by the
Institutional Research Program 011303 "SATGED", Moldova State
University.

\section{Conflict of Interest}

The authors declare that the have not conflict of interest.




\begin{thebibliography}{100}

\bibitem{BLW_2001} Petr W. Bates, Kening Lu and Bixiang Wang,
\newblock Attractors for Lattice Dynamical Systems.
\newblock International Journal of Bifurcation
and Chaos, Vol. 11, No. 1 (2001), pp.143-153.


\bibitem{Che_2015} Cheban D. N.
\newblock{\em Global Attractors of Nonautonomous Dynamical and Control
Systems. 2nd Edition.}
\newblock Interdisciplinary Mathematical Sciences,
vol.18, River Edge, NJ: World Scientific, 2015, xxv+589 pp.


\bibitem{Che_2020} David N. Cheban,
\newblock{\em Nonautonomous Dynamics: \emph{Nonlinear Oscillations and
Global Attractors}.}
\newblock{\em Springer Nature Switzerland AG 2020,} xxii+ 434 pp.

\bibitem{Che_2024} David N. Cheban,
\newblock Monotone Nonautonomous Dynamical Systems.
\newblock{\em Springer Nature Switzerland AG,} 2024, xix+460 pp.


\bibitem{DK_1970}
Daletskii Yu. L. and Krein M. G.,
\newblock {\em Stability of Solutions of Differential Equations in Banach
Space}.
\newblock Moscow, "Nauka", 1970, 534 pp. [English transl., Amer. Math. Soc., Providence, RI
1974.]

\bibitem{HK_2023} Xiaoying Han and Peter Kloeden,
\newblock Dissipative Lattice Dynamical systems.
\newblock World Scientific, Singapoor, 2023, xv+364 pp.

\bibitem{Sel_1971}
Sell G. R.,
\newblock  Lectures on Topological Dynamics and Differential Equations,
{\bf vol.2} of {\it Van Nostrand Reinhold math. studies}.
\newblock Van Nostrand--Reinbold, London, 1971.

\bibitem{sib}
K. S. Sibirsky,
\newblock {\em Introduction to Topological Dynamics.\/}
\newblock Kishinev, RIA AN MSSR, 1970, 144 p. (in Russian). [English
translationn: Introduction to Topological Dynamics. Noordhoff,
Leyden, 1975]

\bibitem{hartman}
P. Hartman, On stability in the large for systems of ordinary differential equations.
Can. J. Math. 13 (1961), 480-492

\end{thebibliography}

\end{document}
