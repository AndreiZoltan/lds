\documentclass[a4paper, 11pt, twoside]{article}

\author{Андрей Султан}
% \title{Топология}
\date{\today}

%%% Работа с русским языком
\usepackage{cmap}					% поиск в PDF                                 UNLOCK
% \usepackage{mathtext} 				% русские буквы в формулах
\usepackage[T2A]{fontenc}
\usepackage[utf8]{inputenc}
% \usepackage[cp1251]{inputenc}
\usepackage[english, russian]{babel}
\usepackage[indentfirst]{titlesec} % абзацпосле первого заголовка
\frenchspacing                      % равные пробелы между словами и предложениями

\usepackage{amssymb,amsmath,amsthm,amsfonts,commath,bm,mathtools}
% \usepackage{icomma}                                    % "Умная" запятая                                 UNLOCK

%%% Свои символы и команды
% \usepackage{centernot} % центрированное зачеркивание символа                                 UNLOCK
% \usepackage{stmaryrd}  % некоторые спецсимволы                                 UNLOCK

\newcommand{\eps}{\ensuremath{\varepsilon}}
\renewcommand{\epsilon}{\ensuremath{\varepsilon}}
\renewcommand{\phi}{\ensuremath{\varphi}}
\renewcommand{\kappa}{\ensuremath{\varkappa}}
\renewcommand{\le}{\ensuremath{\leqslant}}
\renewcommand{\leq}{\ensuremath{\leqslant}}
\renewcommand{\ge}{\ensuremath{\geqslant}}
\renewcommand{\geq}{\ensuremath{\geqslant}}
\renewcommand{\emptyset}{\ensuremath{\varnothing}}

\DeclareMathOperator{\sgn}{sgn}
\DeclareMathOperator{\ke}{Ker}
\DeclareMathOperator{\im}{Im}
\DeclareMathOperator{\re}{Re}
\DeclareMathOperator{\rk}{rk}
\DeclareMathOperator{\pr}{pr}
\DeclareMathOperator{\Span}{Span}
\DeclareMathOperator{\tr}{tr}
\DeclareMathOperator{\PR}{P}
\DeclareMathOperator{\Id}{Id}
\DeclareMathOperator{\Cl}{Cl}

\newcommand{\rom}[1]{%
  \textup{\uppercase\expandafter{\romannumeral#1}}%
}\newcommand{\N}{\mathbb{N}}
\newcommand{\Z}{\mathbb{Z}}
\newcommand{\Q}{\mathbb{Q}}
\newcommand{\R}{\mathbb{R}}
\newcommand{\Cm}{\mathbb{C}}
\newcommand{\F}{\mathbb{F}}
\newcommand{\id}{\mathrm{id}}

\newcommand\quotient[2]{
        \mathchoice
            {% \displaystyle
                \text{\raise1ex\hbox{$#1$}\Big/\lower1ex\hbox{$#2$}}%
            }
            {% \textstyle
                #1\,/\,#2
            }
            {% \scriptstyle
                #1\,/\,#2
            }
            {% \scriptscriptstyle  
                #1\,/\,#2
            }
    }

% \newcommand{\imp}[2]{
% 	(#1\,\,$\ra$\,\,#2)\,\,
% }
% \newcommand{\System}[1]{
% 	\left\{\begin{aligned}#1\end{aligned}\right.
% }
% \newcommand{\Root}[2]{
% 	\left\{\!\sqrt[#1]{#2}\right\}
% }

% \renewcommand\labelitemi{$\triangleright$}


%%% Перенос знаков в формулах (по Львовскому)
\renewcommand{\hm}[1]{#1\nobreak\discretionary{}{\hbox{$\mathsurround=0pt #1$}}{}}

\usepackage[
	backend=biber,
	style=alphabetic,
	sorting=ynt
	]{biblatex}
\usepackage{csquotes}
% \usepackage{natbib}
% \setcitestyle{authoryear,open={[},close={]}}
% \bibliographystyle{apalike}


%%% Работа с картинками
% \usepackage{graphicx}    % Для вставки рисунков                                 UNLOCK
% \setlength\fboxsep{3pt}  % Отступ рамки \fbox{} от рисунка                                 UNLOCK
% \setlength\fboxrule{1pt} % Толщина линий рамки \fbox{}                                 UNLOCK
% \usepackage{wrapfig}     % Обтекание рисунков текстом                                 UNLOCK


%%% Работа с таблицами
% \usepackage{array,tabularx,tabulary,booktabs} % Дополнительная работа с таблицами                                 UNLOCK
% \usepackage{longtable}                        % Длинные таблицы                                 UNLOCK
% \usepackage{multirow}                         % Слияние строк в таблице                                 UNLOCK


%%% Оформление страницы
% \usepackage{extsizes}     % Возможность сделать 14-й шрифт
\usepackage{geometry}     % Простой способ задавать поля
\usepackage{setspace}     % Интерлиньяж
\usepackage{enumitem}     % Настройка окружений itemize и enumerate
\setlist{leftmargin=25pt} % Отступы в itemize и enumerate

\geometry{top=25mm}    % Поля сверху страницы
\geometry{bottom=30mm} % Поля снизу страницы
\geometry{left=20mm}   % Поля слева страницы
\geometry{right=20mm}  % Поля справа страницы


\setlength\parindent{15pt}        % Устанавливает длину красной строки 15pt
% \linespread{1.3}                  % Коэффициент межстрочного интервала
%\setlength{\parskip}{0.5em}      % Вертикальный интервал между абзацами
%\setcounter{secnumdepth}{0}      % Отключение нумерации разделов
%\setcounter{section}{-1}         % Нумерация секций с нуля
% \usepackage{multicol}			  % Для текста в нескольких колонках                                 UNLOCK
% \usepackage{soulutf8}             % Модификаторы начертания                                 UNLOCK


%%% Содержаниие
\usepackage{tocloft}
\tocloftpagestyle{main}
%\setlength{\cftsecnumwidth}{2.3em}
%\renewcommand{\cftsecdotsep}{1}
%\renewcommand{\cftsecpresnum}{\hfill}
%\renewcommand{\cftsecaftersnum}{\quad}


%%% Шаблонная информация для титульного листа
\newcommand{\CourseName}{Название курса поменьше}
\newcommand{\FullCourseNameFirstPart}{\so{БОЛЬШОЕ НАЗВАНИЕ КУРСА}}
\newcommand{\SemesterNumber}{V}
\newcommand{\LecturerInitials}{Иван Иванович Иванов}
\newcommand{\CourseDate}{осень 2022}
\newcommand{\AuthorInitials}{Павел Дуров}
\newcommand{\GithubLink}{https://github.com/daniild71r/lectures_tex_club}


%%% Колонтитулы
%%% Колонтитулы
\usepackage{titleps}
\newpagestyle{main}{
	\setheadrule{0.4pt}
	\sethead
	[\thepage][\hyperlink{intro}{\;содержание}][\thesection\ \sectiontitle]
	{\thesubsection\ \subsectiontitle}{\hyperlink{intro}{\;содержание}}{\thepage}
	% \setfootrule{0.4pt}                       
	% \setfoot{Zol-LAB, \CourseDate}{center}{\thepage} 
}

% \newpagestyle{main}[\small\sffamily]{
% \sethead[\textbf{\thepage}]
% [\textsl{\chaptertitle}]
% [[\toptitlemarks\thesection--\bottitlemarks\thesection]
% {\toptitlemarks\thesection--\bottitlemarks\thesection]}
% {\textsl{\sectiontitle}}
% {\textbf{\thepage}}}

\pagestyle{main} 


%%% Нумерация уравнений
\makeatletter
\def\eqref{\@ifstar\@eqref\@@eqref}
\def\@eqref#1{\textup{\tagform@{\ref*{#1}}}}
\def\@@eqref#1{\textup{\tagform@{\ref{#1}}}}
\makeatother                      % \eqref* без гиперссылки
\numberwithin{equation}{section}  % Нумерация вида (номер_секции).(номер_уравнения)
\mathtoolsset{showonlyrefs=true} % Номера только у формул с \eqref{} в тексте.


%%% Гиперссылки
\usepackage{hyperref}
\usepackage[usenames,dvipsnames,svgnames,table,rgb]{xcolor}
\hypersetup{
	unicode=true,            % русские буквы в раздела PDF
	colorlinks=true,       	 % Цветные ссылки вместо ссылок в рамках
	linkcolor=black!15!blue, % Внутренние ссылки
	citecolor=lonestar,      % Ссылки на библиографию
	filecolor=magenta,       % Ссылки на файлы
	urlcolor=NavyBlue,       % Ссылки на URL
}
\usepackage{colors}
%%% Графика
\usepackage{tikz}        % Графический пакет tikz
\usepackage{tikz-cd}     % Коммутативные диаграммы
\usepackage{tkz-euclide} % Геометрия
\usepackage{stackengine} % Многострочные тексты в картинках
\usetikzlibrary{angles, babel, quotes}


\theoremstyle{plain}
\newtheorem{theorem}{Теорема}[section]
\newtheorem{lemma}[theorem]{Лемма}
\newtheorem{proposition}[theorem]{Предложение}
\newtheorem*{corollary}{Следствие}
\newtheorem*{conjecture}{Гипотеза}
\newtheorem*{criterion}{Критерий}
\newtheorem*{assertion}{Утверждение}

\theoremstyle{definition}
\newtheorem{definition}{Определение}[section]
\newtheorem*{axiom}{Аксиома}
\newtheorem{example}{Пример}[section]
\newtheorem{exercise}[example]{Упражнение}
\newtheorem{problem}{Задача}[section]
\newtheorem*{condition}{Условие}
\newtheorem*{algorithm}{Алгоритм}
\newtheorem*{question}{Вопрос}
\newtheorem*{property}{Свойство}
\newtheorem*{assumption}{Предположение}
\newtheorem*{reminder}{Напоминание}

\theoremstyle{remark}
\newtheorem*{note}{Замечание}
\newtheorem*{solution}{Решение}
\newtheorem*{notation}{Обозначение}
\newtheorem*{summary}{Резюме}
\newtheorem*{acknowledgment}{Благодарность}
\newtheorem*{case}{Случай}
\newtheorem*{conclusion}{Вывод}