\documentclass[10pt,a4paper]{article}
\usepackage{amsfonts}
\usepackage{amsbsy}
\usepackage{amssymb}
\usepackage{amsmath}
\usepackage{amsthm}
\newcommand{\titleart}[1]{\bigskip\begin{center}\large
\textbf{#1}\end{center}\vspace{-6mm}}
\newcommand{\autorart}[1]{\begin{center}\large
\textsf{#1}\end{center} \vspace{-6mm}}
\newcommand{\coord}[1]{\begin{center} \small
\textit{#1}\end{center} \vspace{-6mm}}
\newcommand{\mail}[1]{\begin{center} \small
\texttt{#1}\end{center} \normalsize}
\renewcommand{\thefootnote}{\fnsymbol{footnote}}


\begin{document}


\titleart{ALMOST PERIODIC SOLUTIONS OF LATTICE DYNAMICAL SYSTEMS WITH MONOTONE NONLINEARITY}

% \autorart{David Cheban\index{Cheban D. N.}, Andrei
% Sultan\index{Sultan A. A.} 
\autorart{David Cheban\index{Cheban D. N.},\\
         Andrei Sultan\index{Sultan A. A.}
\footnote{{\it Sultan Andrei}: A. A.
Sultan}}

\coord{Vladimir Andrunachievici Institute of Mathematics and
Computer Science, Moldova State University, Chi\c{s}in\u{a}u,
Republic of Moldova}

\mail{ david.ceban@usm.md, andrew15sultan@gmail.com}

% \addc{Initials. Last name of the first author, Initials. Last name of the second author}{ Title of the abstract}
% \stopcr

\medskip



Let $\mathbb R$ (respectively, $\mathbb Z$) be the set of all real
(respectively, integer) numbers and $l_{2}$ be the Hilbert space of
all sequences $\xi =(\xi_{i})_{i\in\mathbb Z}$ ($\xi_{i}\in
\mathbb R$ for any $i\in \mathbb Z$) with the property
$\sum\limits_{i\in \mathbb Z}|\xi_{i}|^{2}<\infty$ equipped with
the scalar product $\langle \xi,\eta\rangle :=\sum\limits_{i\in
\mathbb Z}\xi_{i}\eta_{i}$. Denote by $C(\mathbb R,l_{2})$ the
family of all continuous functions $\varphi :\mathbb R\to l_{2}$
equipped with the compact-open topology and $(C(\mathbb
R,l_{2}),\mathbb R,\sigma)$ the shift dynamical system on the
space $C(\mathbb R,l_{2})$.

A subset $A\subset \mathbb R$ is called relatively dense if there
exists a positive number $l$ such that $[a,a+l]\bigcap \mathbb R
\not= \emptyset$, \(\forall a \in \mathbb R\).

\textbf{Definition.} A function $\varphi\in C(\mathbb R,l_{2})$ is
said to be almost periodic [1] if for any $\varepsilon
>0$ there exists a relatively dense subset $\mathcal
P(\varepsilon)$ such that
\|\varphi(t+\tau)-\varphi(t)\|_{l_{2}}<\varepsilon
% $|\varphi(t+\tau)-\varphi(t)|_{\ell_2}<\varepsilon$ for any $t\in \mathbb
% R$.

In this talk we study the almost periodic solutions of the systems
\begin{equation}\label{eqI1}
u_{i}'=\nu (u_{i-1}-2u_i+u_{i+1})-\lambda u_{i}+F(u_i)+f_{i}(t)\
(i\in \mathbb Z),
\end{equation}
where $\lambda, \nu >0$, $F\in C(\mathbb R, \mathbb R)$ and $f\in
C(\mathbb R,\ell_{2})$ ($f(t):=(f_{i}(t))_{i\in \mathbb Z}$ for
any $t\in \mathbb R$) is an almost periodic function.

\textbf{Theorem 1.} Suppose that the following conditions hold:
\begin{enumerate}
\item the function $f\in C(\mathbb R,l_{2})$ is almost periodic;
\item the function $F$ possesses the following
properties: \begin{enumerate} \item it is locally Lipschitzian;
\item there exists a positive number $\alpha$ such that $F(s)s\le
-\alpha s^{2}$ for any $s\in \mathbb R$; \item the function $F$ is
monotone [1], i.e., there exists a positive number $\beta$ such
that $(F(x_1)-F(x_2))(x_1-x_2)\le -\beta |x_1-x_2|^{2}$ for any
$x_1,x_2\in \mathbb R$.
\end{enumerate}
\end{enumerate}

Then the equation (\ref{eqI1}) has a unique almost periodic
solution.

\medskip
\textbf{Funding.} This research was supported by the State Program
of the Republic of Moldova "Monotone Nonautonomous Dynamical
Systems (24.80012.5007.20SE)" and partially was supported by the
Institutional Research Program 011303 "SATGED", Moldova State
University.

\medskip
\textbf{ORCID (D. Cheban):} https://orcid.org/0000-0002-2309-3823


\textbf{ORCID (A. Sultan):} https://orcid.org/0009-0003-9785-9291


\medskip
{\small \centerline{\bf References:}
\smallskip
\noindent 1. David N. Cheban. {\it Monotone Nonautonomous
Dynamical Systems.} Springer Nature Switzerland AG, 2024.

\end{document}
